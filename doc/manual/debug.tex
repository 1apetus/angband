\section{Debug Command Descriptions --- debug.txt}
\subsection{Item Creation}\index{creation!item}
\begin{description}
\item[Create an object (c)]
    Provides a menu to let you create any object, and drops it on the
    floor.

\item[Create an artifact (C)]
    Prompts you for the name of an artifact, then drops that artifact
    nearby. You must give the name exactly as in artifact.txt. You may
    optionally give a command-count, in which case this command drops the
    artifact with that number nearby instead of prompting you for a name.

\item[Create a good object (g)]
    Creates a good object and places it nearby. If you provide a command-
    count, creates that many good items.

\item[Create a very good object (v)]
    Creates a very good (``excellent'') object and places it nearby. If you
    provide a command-count, creates that many very good items.

\item[Play with an object (o)]
    Lets you modify an object by randomly rerolling it as a normal, good,
    or excellent object, or lets you modify it directly, tweaking the pval
    and combat values.

\item[Test kind (V)]
    Requires a command-count. For the tval given by command-count, creates
    one object of each sval and drops it nearby.
\end{description}

\subsection{Detection / Information}
\begin{description}
\item[Detect all (d)]
    Detects all traps, doors, stairs, treasure, and monsters nearby.

\item[Identify (i)]
    Fully identifies an object.

\item[Magic Mapping (m)]
    Maps the nearby dungeon.

\item[Self-knowledge (k)]
    Grants you self-knowledge, as the potion of the same name.

\item[Learn about objects (l)]
    Requires a command-count. Makes you ``aware'' of all items with level
    less than or equal to the command-count.

\item[Monster recall (r)]
    Gives you full monster recall on all monsters or on a chosen monster.

\item[Unhide monsters (u)]
    Reveals all monsters whose distance to the character is at most 255.
    If given a command-count, uses that distance instead of 255.

\item[Wizard-light the level (w)]
    Lights the entire level, as the Potion of Enlightenment.

\item[Create spoilers (")]
    Lets you create a spoiler file for objects or monsters.
\end{description}

\subsection{Teleportation}
\begin{description}
\item[Teleport level (j)]
    Allows you to teleport to any dungeon level instantly.

\item[Phase Door (p)]
    Teleports you up to 10 spaces away.

\item[Teleport (t)]
    Teleports you up to 100 spaces away.

\item[Teleport to target (b)]
    Teleports you to the last space you targeted (or close to it, if the
    space is occupied).
\end{description}

\subsection{Character Improvement}
\begin{description}
\item[Cure all maladies (a)]
    Removes all curses, restores all stats, xp, hp, and sp, cures
    all bad effects, and satisfies your hunger.
\item[Advance the character (A)]
    Advances your character to level 50, maxes all stats, and gives you
a million gold.
\item[Edit character (e)]
    Lets you specify your base stats, xp, and gold.
\item[Increase experience (x)]
    Doubles your current experience and adds 1. If given a command-count,
increases your experience by that much instead.

\item[Rerate hitpoints (h)]
    Rerates your hitpoints.
\end{description}

\subsection{Monsters}
\begin{description}
\item[Summon monster (n)]
    Prompts you for the name of a monster, then summons that monster
    nearby. You must give the name exactly as in monster.txt. You may
    optionally give a command-count, in which case this command summons the
    monster with that number nearby instead of prompting you for a name.

\item[Summon random monster (s)]
    Summons a random monster next to you. If given a command-count,
    summons that many monsters instead.

\item[Zap monsters (z)]
    Deletes all monsters in sight. If given a command-count, deletes all
monsters whose distance to the character is at most the command-count
instead.
\end{description}

\subsubsection{Miscellaneous}
\begin{description}
\item[Create a trap (T)]
Creates a random trap on your square.
\end{description}

\subsubsection{Undocumented}
\begin{description}
\item[Query the dungeon (q)]
???

\item[Collect stats (f)]
???

\item[Ben hack (\_)]
    ???
\end{description}
