\section{Birth --- birth.txt}
\subsection{Creating a Character}
\paragraph{}
Angband is a roleplaying game, in which you, the player, control a
character in the world of Angband. Perhaps the most important thing you
control is the birth of your character, in which you choose or allow to
be chosen various attributes that will affect the future life of your
character.

\paragraph{}
At the character creation screen you will be prompted to select the sex,
race and class of your character. You also have the option to change
the 'birth options' at this time. These need to be set at the character
creation menu and cannot be altered later in the game. They are
discussed with the rest of the options in the ``options.txt'' help file.


\subsection{Character Characteristics}
\paragraph{}
Each character has three primary attributes: sex, race, and class. These
are chosen at the beginning and which will stay fixed for the entire
life of that character. The sex of your character is purely for
flavour, but the race and class have many effects which are discussed in
detail below.

\paragraph{}
Each character has a few secondary attributes: height, weight, social
class, and background history. These are randomly determined, but are
affected by the sex and race of the character. In general, these
attributes are only used to provide flavour to the character, to assist
in the role playing, but they do have a few minor effects on the game.
For example, background history affects social class, which affects the
starting money. (Not a lot, in the case of some races: for instance,
Half-Trolls are always going to be the scum of society, even if their
father was the Clan Chief.)

\paragraph{}
Each character also has six primary "stats": strength, intelligence,
wisdom, dexterity, constitution, and charisma, which modify the
abilities of the character in a variety of ways. Every stat has a
numerical value, ranging from a minimum of 3, up to a normal maximum of
18, and even higher, into the "percentile" range, represented as "18/01"
through "18/100": this is the maximum that can be achieved
intrinsically, for any given stat. These stats can be modified further
by equipment, race and class bonuses up to a maximum of 18/220.

\paragraph{}
Each character also has several primary "skills": disarming, magic
devices, saving throws, stealth, searching ability, searching frequency,
fighting skill, and shooting skill, which are derived from the
character's race, class, level, stats, and current equipment. These
skills have rather obvious effects, but will be described more
completely below.

\paragraph{}
Each character may have one or more "racially intrinsic skills", based
on the race of the character. These may include special resistances, or
abilities such as infravision.

\paragraph{}
Each character has a number of "experience points", which increases as
the character defeats monsters and attempts new spells and uses new
items. Characters also have a level, which is based on experience. The
amount of experience required to gain a new level is dependent on the
character race and class. Races and classes with more intrinsic
benefits require more experience to gain levels. As the experience
rises, so does the level, and as the level rises, certain other
abilities and characteristics rise as well. All characters start at 0
experience and at the first level.

\paragraph{}
Each character has some gold, which can be used to buy items from the
shops in the town, and which can be obtained not only from selling items
to the shops, but also by taking it from dead monsters and by finding it
in the dungeon. Each character starts out with some gold, the amount of
which is based on the character's social class (higher is better),
charisma (higher is better), and stats (less powerful characters start
with more gold). Each character also starts out with a few useful
items, which may be kept, or sold to a shop-keeper for more gold.
However, especially valuable items will never sell for the full price,
as each shopkeeper has a maximum that he is prepared to pay for any
item. The more generous shopkeepers may buy your items for up to 30,000
gold pieces: but some are really stingy, and will pay no more than
5,000.

\paragraph{}
Each character has an "armour class", abbreviated to AC, representing
how well the character can avoid damage. The armour class is affected
by dexterity and equipment, so the concept includes both avoiding blows
and being able to take blows without being hurt. Armour class on
equipment is always denoted in [square brackets], usually as a figure of
[X,+Y] where X is the intrinsic AC of the armour in question, and Y is
the magical bonus to armour class provided by that item.

\paragraph{}
Each character has "hit points", or hp, representing how much damage the
character can sustain before he dies. How many hit points a character
has is determined by race, class, level and constitution, as follows:
each race has a basic "hit dice" number --- for instance, a Dwarf's basic
hit die is 11, while a hobbit's is 7. This is modified by class: for
instance, a warrior gets a +9 bonus to the hit die, while a mage gets no
bonus and a priest +2, so a dwarven warrior's total hit die will be
(11+9)=20 - meaning that he gets between 1 and 20 hit points per level.
If he were a priest, his hit dice would be (11+2)=13, and get between 1
and 13 hit points per level. The hobbit mage would get only 1-7 hps per
level. (All characters get the maximum at first level: thereafter it is
a random roll of 1dX where X is the hit die number, when the character
goes up in level.) This is further modified by constitution - a
character with high constitution will get a flat bonus of a certain
number of hit points per level (recalculated right back to level 1: if
you gain an extra hit point per level, and you are 42nd level, you will
suddenly be 42 hit points better off.)

\paragraph{}
Each character has spell points, or mana, which limits how many spells
(or prayers) a character can cast (or pray). The maximum number of
spell points is derived from your class, level, and intelligence (for
spells) or wisdom (for prayers), and you can never have more spell
points than the maximum. Spell points may be regained by resting, or by
magical means. Warriors never have any spell points. If a character
gains enough wisdom or intelligence to get more spell points, the result
is calculated right back to first level, just as with constitution and
hit points.

\paragraph{}
Lastly, each character has a base speed. Speed determines the amount of
"energy" your character acquires in the game, and therefore how often
you can take actions which use up energy (like moving or attacking).
All beginning characters move at normal speed and the only way to
increase speed is by magical means and equipment bonuses. Characters
who are carrying too much weight will move more slowly. Extra speed is
one of the most important boons in the game and therefore one of the
rarest and most sought after.

\subsection{Races}
\paragraph{}
There are eleven different races that you can choose to play in Angband.
Each race has its own adjustments to a character's stats and abilities.
Most races also have intrinsic abilities. The bonuses to statistics and
the experience penalty will be displayed next to the races as you move
to select one.
\begin{description}
\item[Human] The human is the base character. All other races are compared
     to them. Humans are average at everything and tend to go
     up levels faster than any other race due to their shorter life
     spans. No racial adjustments or intrinsics occur to human
     characters. Humans do not have any infravision.

\item[Half-Elf] Half-elves tend to be smarter and faster than a human, but
     not as wise or strong. Half-elves are slightly better at magic,
     disarming, saving throws, stealth, archery and searching, but
     they are not as good at hand-to-hand combat. Half-elves are
     immune to dexterity draining, and have weak infravision.

\item[Elf] Elves are better magicians than humans, but not as good at
     fighting. They tend to be smarter and faster than humans,
     though not as wise or strong. Elves are better at searching,
     disarming, perception, stealth, archery and magic, but they are
     not as good at hand-to-hand combat. They are resistant to
     attacks involving bright light, are immune to dexterity
     draining, and have fair infravision.

\item[Hobbit] Hobbits, or Halflings, are very good at shooting, throwing, and
     have good saving throws. They also are very good at search-
     ing, disarming, perception, and stealth; so they make excel-
     lent rogues, but prefer to be called burglars. They will be
     much weaker than humans, and not good at melee fighting.
     Halflings have fair infravision. They have a strong hold
     on their life force, and are thus resistant to life
     draining. Hobbits are very partial to mushrooms and can
     identify them when found.

\item[Gnome] Gnomes are smaller than dwarves but larger than halflings.
     They, like the halflings, live in the earth in burrow-like
     homes. Gnomes make excellent mages, and have very good saving
     throws. They are good at searching, disarming, perception,
     and stealth. They have lower strength than humans so they
     are not very good at fighting with hand weapons. Gnomes have
     good infravision, so they can detect warm-blooded creatures
     up to 40 feet away. Gnomes are intrinsically protected against
     paralysis and some slowing effects. Gnomes are excellent at
     using wands and staves and can identify them when found.

\item[Dwarf] Dwarves are the headstrong miners and fighters of legend.
     Since dungeons are the natural home of a dwarf, they are
     excellent choices for a warrior or priest --- or indeed, that
     combination of the two, the paladin. Dwarves tend to be
     stronger and tougher but slower and less intelligent than
     humans. Because they are so headstrong and are somewhat wise,
     they resist spells which are cast on them. Dwarves also have
     excellent infravision. They can never be blinded. Dwarves
     are excellent at digging, and can sense nearby buried treasure.
     They have one big drawback, though. Dwarves are loudmouthed and
     proud, singing in loud voices, arguing with themselves for no
     good reason, screaming out challenges at imagined foes. In
     other words, dwarves have a miserable stealth.

\item[Half-Orc] Half-Orcs make excellent warriors and decent priests, but
     are terrible at magic. They are as bad as dwarves at stealth,
     and horrible at searching, disarming, and perception.
     Half-Orcs are, let's face it, ugly. They tend to pay more for
     goods in town. Half-Orcs do make good warriors and rogues,
     for the simple reason that Half-Orcs tend to have great
     constitution and lots of hit points. Because of their
     preference to living underground to on the surface, half-orcs
     resist darkness attacks. They have fair infravision.

\item[Half-Troll] Half-Trolls are incredibly strong, and have more hit points
     than any other character race. They are also very stupid and
     slow. They will make great warriors and iffy priests. They
     are bad at searching, disarming, perception, and stealth.
     They are so ugly that a Half-Orc grimaces in their presence.
     Half-trolls always have their strength sustained, and they
     regenerate quickly like other trolls. Unfortunately, this
     regeneration also requires them to eat more food than other
     races. They have fair infravision.

\item[D\'{u}nadan] D\'{u}nedain are a race of hardy men from the West. This elder
     race surpasses human abilities in every field, especially
     constitution. However, being men of the world, very little is
     new to them, and levels are very hard to gain... their
     hardiness ensures that their constitution cannot be reduced.
     They have no infravision.

\item[High-Elf] High-Elves are descended from those among the Elves who heard
     and answered the call from the Valar at the very beginning of
     time, before the sun and moon were made, and lived in the
     Blessed Realm for many thousands of years before returning to
     mortal lands. Because of this, they are far superior in terms
     of abilities when compared to their lesser Elven kindred. They
     can also see into the invisible world of ghosts and wraiths.
     However, they find new experience even harder to come by than
     D\'{u}nedain. Like normal Elves, they resist attacks involving
     bright light. They have good infravision and can even see
     cold-blooded invisible creatures.

\item[Kobold] Kobolds are a race of small dog-headed humanoids that dwell
     underground. They have excellent infravision, and are
     intrinsically resistant to poisons of all kinds. They have
     a good dexterity and constitution. However, they are
     weaker than humans, and also not noted for great
     intelligence. Furthermore, they are ugly, and not
     trusted in town. And while their constitution makes them
     tough, it still cannot prevent the fact that they are not the
     biggest of creatures, and have few hit points.
\end{description}

\subsection{Classes}
\paragraph{}Once a race has been chosen, you will need to pick a class. The class
is the character's occupation and determines stat bonuses, abilities,
hit dice, and what spells (if any) the character can learn.
\begin{description}
\item[Warrior] A Warrior is a hack-and-slash character, who solves most of
     his problems by cutting them to pieces, but will occasionally
     fall back on the help of a magical device. His prime stats
     are strength, constitution, and dexterity, and he will strike
     more blows with melee weapons than any other class. A Warrior
     will be excellent at fighting, shooting and throwing, but bad
     at most other skills. A warrior has bad stealth and cannot
     learn any spells.

 \item[Mage] A Mage must live by his wits. He cannot hope to simply hack
     his way through the dungeon, and so must therefore use his
     magic to defeat, deceive, confuse, and escape. A mage is
     not really complete without an assortment of magical devices
     to use in addition to his spells. He can master the higher
     level magical devices far easier than anyone else, and has
     the excellent saving throws to resist effects of spells cast at
     him. However, he is incredibly weak, getting few hit dice
     and suffering strong penalties to strength and constitution.
     Intelligence is his primary stat and at high levels he can
     cast many spells without a chance of failure. There is no
     rule that says a mage cannot become a good fighter, but
     spells are his true realm and he will get fewer blows with
     melee weapons than other classes.

 \item[Priest] A Priest is a character of holy devotion. They explore the
     dungeon only to destroy the evil that lurks within, and if
     treasure just happens to fall into their packs, well, so
     much more to the glory of their church! Priests receive
     their spells from a deity, and therefore do not choose which
     spells they will learn. They are familiar with magical
     devices, preferring to call them "instruments of God", but
     are not as good as a mage in their use. Priests have great
     saving throws, and make decent fighters, but they are not
     as good at using sharp weapons (swords and polearms) owing to
     the Church's strictures about shedding blood, so they are
     better off using blunt weapons such as maces. Wisdom is the
     priest's primary stat and at high enough levels they can cast
     many prayers without a chance of failure. Priests have poor
     stealth.

 \item[Rogue] A Rogue is a character that prefers to live by his cunning,
     but is capable of fighting his way out of a tight spot. He
     is the master of traps and locks, no device being impossible
     for him to overcome. A rogue has a high stealth allowing
     him to sneak around many creatures without having to fight,
     or sneak up and get the first blow. A rogue's perception is
     higher than any other class, and many times he will notice a
     trap or secret door before having to search. A rogue is
     better than warriors or paladins with magical devices, but
     still cannot rely on their performance. Rogues can also
     learn a few spells, but not the powerful offensive spells
     mages can use, and there will always be a chance of failure
     even with the simplest spells. A rogue's primary stats are
     dexterity and intelligence.

 \item[Ranger] A Ranger is a warrior/mage and a very powerful class. He is
     a good fighter, and the best of all the classes with missile
     weapons, especially bows. The ranger learns spells much more
     slowly than a mage, but is capable of learning all but the
     most powerful spells. Because a ranger is really a dual
     class character, more experience is required for him to
     advance. A ranger has good stealth, good perception, good
     searching, a good saving throw, and is good with magical
     devices. His primary stats are strength, intelligence and
     dexterity.

 \item[Paladin] A Paladin is a warrior/priest. He is a very good fighter,
     second only to the warrior class, but not very good at missile
     weapons. He receives prayers at a slower pace then the
     priest, but can use all but the most powerful prayers.
     Because a paladin is really a dual class character, it
     requires more experience to advance him. A paladin lacks
     much in the way of abilities. He is poor at stealth, perception,
     searching, and magical devices. He has a decent
     saving throw due to his divine alliance. His primary stats
     are strength and wisdom.
\end{description} 

\subsection{Stats}
\paragraph{}
After gender, race and class are selected, you will be able to decide
what stat levels your character will have, by allocating a finite number
of "points" between the six statistics. These points can be allocated
by selection or with a random roller (as described below in the "Stat
Rollers" section). Each race/class combination also has a recommended
default setting for these statistics. Statistics can be permanently
raised by various potions in the dungeon up to 18/100. They can also be
temporarily drained by some monster attacks.

\begin{description}
\item[Strength]
  Strength is important in fighting with weapons and in melee
  combat. A high strength can improve your chances of hitting
  as well as the amount of damage done with each hit. Characters
  with low strengths may receive penalties. Strength
  raises the amount of weight you can carry before being slowed.
  It also allows you to get extra blows with heavier weapons.
  Strength is one of the most important stats in the beginning
  of the game.
 
\item[Intelligence]
  Intelligence affects the spellcasting abilities of mage-like
  spellcasters: mages, rangers, and rogues. Intelligence
  will affect the number of spells you may learn each level as
  well as the number of spell points you receive. Intelligence
  is the most important stat for mages. A high
  intelligence may also improve your chances of successfully
  casting a spell. You cannot learn spells if your intelligence
  is 7 or lower. A good intelligence can also help with using
  magic devices, picking locks, and disarming traps.
 
\item[Wisdom]
  The primary function of wisdom is to determine the ability
  of a priest or paladin to use prayers, just like intelligence
  affects spellcasting. Again, high wisdom will increase the
  number of mana points you have and increase the number of
  prayers you can learn each level, while improving the chance
  that a prayer will be successful. A good wisdom increases
  your saving throw, thereby improving your chances of resisting
  magical spells cast upon you by monsters.
 
\item[Dexterity]
  Dexterity is a combination of agility and quickness. A high
  dexterity may allow a character to get multiple blows with
  lighter weapons. Dexterity also increases a character's
  chances of hitting with any weapon and of dodging blows from
  enemies. Dexterity is also useful in picking locks, disarming
  traps, and protecting yourself from some of the thieves that
  inhabit the dungeons. Indeed, if the character has a high
  enough dexterity, thieves will never be successful in stealing
  from him.
 
\item[Constitution]
  Constitution is a character's ability to resist damage to his
  body, and to recover from damage received. Therefore a
  character with a high constitution will receive more hit
  points and also recover them faster while resting. Constitution
  is less important in the beginning of the game, but will be the
  most important stat at the end.
 
\item[Charisma]
  Charisma represents a character's personality and physical
  appearance. A character with a high charisma will receive
  better prices from store owners, whereas a character with a
  very low charisma may be robbed blind.
\end{description}

\subsection{Skills}
\paragraph{}
Characters possess some different skills which can help them to survive.
The starting skill levels of a character are based upon race and class.
Skill levels may be adjusted by high or low stats, and may increase with
the level of the character.

\begin{description}
\item[Infravision]
  Infravision is the ability to see heat sources. Since most
  of the dungeon is cool or cold, infravision will not allow
  the player to see walls and objects. Infravision will allow
  a character to see any warm-blooded creatures up to a certain
  distance. This ability works equally well with or without a light
  source. However, some of Angband's creatures are 
  cold-blooded, and will not be detected unless lit up by a
  light source. All non-human races have innate infravision.
  Humans (including D\'{u}nedain) cannot gain infravision unless it
  is magically enhanced. Infravision does not increase with
  character level, and is purely dependent on race and on
  magical equipment.
 
\item[Fighting]
  Fighting is the ability to hit and do damage with weapons or
  fists. Normally a character gets a single blow from any
  weapon, but if his dexterity and strength are high enough,
  he may receive more blows with lighter weapons. Strength and
  dexterity both modify the ability to hit an opponent. This
  skill increases with the level of the character. Inspecting a
  weapon will show you how quickly you can attack with it.
 
\item[Shooting Ability]
  Using ranged missile weapons (and throwing objects) is
  included in this skill. Different stats apply to different
  weapons, but this ability may modify the distance an object
  is thrown/fired, the amount of damage done, and the ability
  to hit a creature. This skill increases with the level of
  the character.
 
\item[Saving Throws]
  A Saving Throw is the ability of a character to resist the
  effects of a spell cast on him by another person/creature.
  This does not include spells cast on the player by his own
  stupidity, such as quaffing a nasty potion. This ability
  increases with the level of the character, but then most
  high level creatures are better at casting spells, so it
  tends to even out. A high wisdom also increases this ability.
  It is possible to get 100% saving throw, making you
  immune to many attacks.
 
\item[Stealth]
  The ability to move silently about is very useful. Characters
  with good stealth can usually surprise their opponents,
  gaining the first blow. Also, creatures may fail to notice
  a stealthy character entirely, allowing a player to avoid
  certain fights. This skill is based entirely upon race and
  class, and will never improve unless magically enhanced.
 
\item[Disarming]
  Disarming is the ability to remove traps safely, and also
  includes picking locks on traps and doors. A successful
  disarming or lock picking will gain the character a small
  amount of experience. A trap must be found before it can
  be disarmed. Dexterity and intelligence both modify the
  ability to disarm, and this ability increases with the
  level of the character.
 
\item[Magical Devices]
  Using a magical device such as a wand or staff requires
  experience and knowledge. Spell users such as mages and
  priests are therefore much better at using a magical device
  than say a warrior. This skill is modified by intelligence,
  and increases with the level of the character.

\item[Searching Frequency (Perception)]
  Perception is the ability to notice something without
  actively seeking it out. This skill is based entirely upon
  race and class, and will never improve unless magically
  enhanced.

\item[Searching Ability (Searching)]
  To search is to actively look for secret doors, floor traps,
  and traps on chests. Rogues are the best at searching, but
  mages, rangers, and priests are also good at it. This skill
  is based entirely upon race and class, and will never
  improve unless magically enhanced.
\end{description}

 
\subsection{Stat Bonus Tables}
\paragraph{}
Stats, hit dice, infravision and experience point modifications due to
race and class are listed in the following table. To get the total hit
dice and XP modifier, add the "race" and "class" numbers: for instance,
a Dwarf Priest has a hit die of 11+2=13 (i.e. he will get 1d13 hit
points per level, adjusted for constitution) and an XP modifier of
20+20=40\%.

\paragraph{}
\begin{tabular}{|l|c|c|c|c|c|c|c|c|c|}
\hline
       & STR & INT & WIS & DEX & CON & CHR & HD & XP/lvl &
       Infra\\
\hline
Human      & 0  & 0  & 0  & 0  & 0  & 0  & 10 & +0\%   & None\\
Half-Elf   & -1 & +1 & 0  & +1 & -1 & +1 & 9  & +10\%  & 20'\\
Elf        & -1 & +2 & +1 & +1 & -2 & +1 & 8  & +20\%  & 30'\\
Hobbit     & -2 & +2 & +1 & +3 & +2 & +1 & 7  & +10\%  & 40'\\
Gnome      & -1 & +2 & 0  & +2 & +1 & -2 & 8  & +25\%  & 40'\\
Dwarf      & +2 & -3 & +2 & -2 & +2 & -3 & 11 & +20\%  & 50'\\
Half-Orc   & +2 & -1 & 0  & 0  & +1 & -4 & 10 & +10\%  & 30'\\
Half-Troll & +4 & -4 & -2 & -4 & +3 & -6 & 12 & +20\%  & 30'\\
D\'{u}nadan& +1 & +2 & +2 & +2 & +3 & +2 & 10 & +80\%  & None\\
High-Elf   & +1 & +3 & -1 & +3 & +1 & +5 & 10 & +100\% & 40'\\
Kobold     & -1 & -1 & 0  & +2 & +2 & -2 & 8  & +15\%  & 50'\\
\hline
\end{tabular}

\paragraph{}
\begin{tabular}{|l|c|c|c|c|c|c|c|c|}
\hline
   & STR & INT & WIS & DEX & CON & CHR & HD & XP/level \\
\hline
Warrior & +5 & -2 & -2 & +2 & +2 & -1 & 9 & +0\%\\
Mage    & -5 & +3 & 0  & +1 & -2 & +1 & 0 & +30\%\\
Priest  & -1 & -3 & +3 & -1 & 0  & +2 & 2 & +20\%\\
Rogue   & +2 & +1 & -2 & +3 & +1 & -1 & 6 & +25\%\\
Ranger  & +2 & +2 & 0  & +1 & +1 & +1 & 4 & +30\%\\
Paladin & +3 & -3 & +1 & 0  & +2 & +2 & 6 & +35\%\\
\hline
\end{tabular}

\subsection{Ability Tables}
\paragraph{}
\begin{tabular}{|l|c|c|c|c|c|c|c|c|c|}
\hline
        & Disarm & Device & Save & Stealth & Search & Percep & Fight &
        Bows\\
\hline
Human      & 0  &  0 & 0  &  0 &  0 & 10 &  0  &  0\\
Half-Elf   & 2  &  3 & 3  &  1 &  6 & 11 & -1  &  5\\
Elf        & 5  &  6 & 6  &  1 &  8 & 12 & -5  & 15\\
Hobbit     & 15 & 18 & 18 &  4 & 12 & 15 & -10 & 20\\
Gnome      & 10 & 12 & 12 &  3 &  6 & 13 & -8  & 12\\
Dwarf      & 2  &  9 & 9  & -1 &  7 & 10 & 15  &  0\\
Half-Orc   & -3 & -3 & -3 & -1 &  0 &  7 & 12  & -5\\
Half-Troll & -5 & -8 & -8 & -2 & -1 &  5 & 20  & -10\\
D\'{u}nadan& 4  &  5 & 5  &  2 &  3 & 13 & 15  & 10\\
High-Elf   & 4  & 20 & 20 &  3 &  3 & 14 & 10  & 25\\
Kobold     & 10 &  5 & 0  &  4 & 15 & 15 & -5  & 10\\
\hline
\end{tabular}

\paragraph{}.\\
\begin{tabular}{|l|c|c|c|c|c|c|c|c|c|}
\hline
    & Disarm & Device & Save & Stealth & Search & Percep & Fight &
    Bows\\
\hline
Warrior&25(+10) & 18(+7)  & 18(+10) & 1(+0) & 14(+0) & 7(+0)  & 70(+45) & 55(+45)\\
Mage   &30(+7)  & 36(+13) & 30(+9)  & 2(+0) & 16(+0) & 20(+0) & 34(+15) & 20(+15)\\
Priest &25(+7)  & 30(+10) & 32(+12) & 2(+0) & 16(+0) & 8(+0)  & 48(+20)
& 35(+20)\\
Rogue  &45(+15) & 32(+10) & 28(+10) & 5(+0) & 32(+0) & 24(+0) & 60(+40)
& 66(+30)\\
Ranger &30(+8)  & 32(+10) & 28(+10) & 3(+0) & 24(+0) & 16(+0) & 56(+30)
& 72(+45)\\
Paladin&20(+7)  & 25(+10) & 25(+11) & 1(+0) & 12(+0) & 2(+0)  & 68(+35)
& 40(+30)\\
\hline
\end{tabular}

\paragraph{}
For character classes, there are two figures: the first figure is the
base level of the skill, while the second figure (in parentheses) is
the bonus that the character gains to this skill every ten levels. So,
to find out the total skill value of any character's skills, add the
race value to the class value, and then the bonus once for every ten
levels that the character has.

\paragraph{}
Please note, however, that these numbers are only good for comparing
characters to each other in the absence of other bonuses from high stats
(strength bonus to-dam, dex bonus to-hit, wisdom bonus to saving throw,
intelligence bonus to magical device usage, etc.) or wearing magical
items.


\subsection{Stat rollers}
\paragraph{}
There are currently two different ways to determine the starting stats of
your character - you can choose which one to use from the birth screen.
\begin{description}
\item[Point-based] The point-based method allows you to "buy"
    improvements to your basic stats by "spending" points on them. You
    have a fixed number of points to spend, and making small changes to
    a stat costs proportionally less than making large changes.  Any
    unspent points are converted into your starting money that you can
    use to buy equipment at the start of the game.

    On selecting this option, you will find that the points have
    already been assigned to default recommended values. These
    represent an expert's opinion for the ideal point spending.
    However, you are free to reallocate them as you wish.

    This is the recommended birth method.

\item[Standard roller] The standard roller simply rolls three six-sided dice for each
    stat, leaving everything to chance. You can press 'r' to re-roll
    the dice, or simply accept what luck has offered.
\end{description}

\subsection{Character Name}
\paragraph{}
Once you have accepted a character you will asked to provide
a name for the character. In general, the actual choice of a name is not
important, but do keep in mind that it may have some effect on the game
itself. For example, on some machines, the character name determines the
filename that will be used to save the character to disk. The character
name is used on the high score list.

\paragraph{}
You can play a dynasty of characters. If you use a roman numeral at the
end of your character name (like "Fred I" or "Pimplesnarg XVI"), the game
will automatically increment the numeral each time you die.


