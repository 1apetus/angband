\section{The Dungeon}
\paragraph{}After you have created your character, you will begin your
Angband adventure.  Symbols appearing on your screen will represent the
dungeon's walls, floor, objects, features, and creatures lurking about.
In order to direct your character through his adventure, you will enter
single character commands (see ``commands.txt'').

\subsection{Symbols On Your Map}
\paragraph{}Symbols on your map can be broken down into three categories: Features of
the dungeon such as walls, floor, doors, and traps; Objects which can be
picked up such as treasure, weapons, magical devices, etc; and creatures
which may or may not move about the dungeon, but are mostly harmful to your
character's well being.

\paragraph{}Some symbols are used to represent more than one type of entity, and some
symbols are used to represent entities in more than one category. The "\@"
symbol (by default) is used to represent the character.

\paragraph{}It will not be necessary to remember all of the symbols and their meanings.
The "slash" command ("/") will identify any character appearing on your map
(see "commands.txt").

\paragraph{}Note that you can use a ``user pref file'' to change any of these symbols to
something you are more comfortable with.
   
\paragraph{}
\begin{tabular}{|ll|ll|}
\hline
\multicolumn{4}{|c|}{Features that do not block line of sight}\\
\hline
.&A floor space&1&Entrance to General Store\\
.&A trap (hidden)&2&Entrance to Armoury\\
\^{}&A trap (known)&3&Entrance to Weapon Smith\\
;&A glyph of warding&4&Entrance to Temple\\
'&An open door&5&Entrance to Alchemy Shop\\
'&A broken door&6&Entrance to Magic Shop\\
<&A staircase up&7&Entrance to the Black Market\\
>&A staircase down&8&Entrance to your Home\\
\hline
\end{tabular}

\paragraph{}
\begin{tabular}{|ll|ll|}
\hline
\multicolumn{4}{|c|}{Features that block line of sight}\\
\hline
\#&A secret door&\#&A wall\\
+&A closed door&\%&A mineral vein\\
+&A locked door&*&A mineral vein with treasure\\
+&A jammed door&:&A pile of rubble\\
\hline
\end{tabular}

\paragraph{}
\begin{tabular}{|ll|ll|}
\hline
\multicolumn{4}{|c|}{Objects}\\
\hline
!&A potion (or flask)&/&A pole-arm\\
?&A scroll (or book)&$|$&An edged weapon\\
,&A mushroom (or food)&\textbackslash &A hafted weapon\\
-&A wand or rod&\}&A sling, bow, or x-bow\\
\_&A staff&\{&A shot, arrow, or bolt\\
=&A ring&(&Soft armour\\
"&An amulet&[&Hard armour\\
\$&Gold or gems&]&Misc. armour\\
$\sim$&Lights, Tools, Chests, etc&)&A shield\\
$\sim$&Junk, Sticks, Skeletons, etc&\&&(unused)\\
\hline
\end{tabular}

\paragraph{}
\begin{tabular}{|ll|ll|}
\hline
\multicolumn{4}{|c|}{Monsters}\\
\hline
\$&Creeping Coins&,&Mushroom Patch\\
a&Giant Ant&A&Angelic being\\
b&Giant Bat&B&Bird\\
c&Giant Centipede&C&Canine (Dog)\\
d&Dragon&D&Ancient Dragon\\
e&Floating Eye&E&Elemental\\
f&Feline (Cat)&F&Dragon Fly\\
g&Golem&G&Ghost\\
h&Humanoids&H&Hybrid\\
i&Icky-Thing&I&Insect\\
j&Jelly&J&Snake\\
k&Kobold&K&Killer Beetle\\
l&Giant Louse&L&Lich\\
m&Mold&M&Multi-Headed Hydra\\
n&Naga&N&(unused)\\
o&Orc&O&Ogre\\
p&Human ``person''&P&Giant ``person''\\
q&Quadruped&Q&Quylthulg (Pulsing Flesh Mound)\\
r&Rodent&R&Reptile/Amphibian\\
s&Skeleton&S&Spider/Scorpion/Tick\\
t&Townsperson&T&Troll\\
u&Minor Demon&U&Major Demon\\
v&Vortex&V&Vampire\\
w&Worm or Worm Mass&W&Wight/Wraith\\
x&(unused)&X&Xorn/Xaren\\
y&Yeek&Y&Yeti\\
z&Zombie/Mummy&Z&Zephyr Hound\\
\end{tabular}

\subsection{The Town Level}
\paragraph{}The town level is where you will begin your adventure. The
town consists of eight buildings (each with an entrance), some
townspeople, and a wall which surrounds the town. The first time you are
in town it will be daytime, but note that the sun rises and falls
(rather instantly) as time passes.


\subsubsection{Townspeople}
\paragraph{}The town contains many different
kinds of people. There are the street urchins, young children who will
mob an adventurer for money, and seem to come out of the woodwork when
excited. Blubbering idiots are a constant annoyance, but not harmful.
Public drunks wander about the town singing, and are of no threat to
anyone. Sneaky rogues hang about watching for a likely victim to mug.
And finally, no town would be complete without a swarm of half drunk
warriors, who take offense or become annoyed just for the fun of it.
(There are assumed to be other people in the town, but they are not
represented on the screen as they do not interact with the player in any
way.)

\paragraph{}Most of the townspeople should be avoided by the largest
possible distance when you wander from store to store. Fights will break
out, though, so be prepared. Since your character grew up in this world
of intrigue, no experience is awarded for killing the town inhabitants,
though you may acquire treasure.

\subsubsection{Town Buildings}
\paragraph{}Your character will begin his adventure with some basic supplies, and some
extra gold with which to purchase more supplies at the town stores.

\paragraph{}You may enter any open store to buy or sell items of the appropriate type.
The price that the shopkeeper requests is dependent on the price of the
item and the player's charisma. Stores also have a maximum value, they will
not pay more than that for any item, regardless of how much it is actually
worth.

\paragraph{}Once inside a store, you will see the name and race of the store owner, the
name of the store, the maximum amount of cash that the store owner will pay
for any one item, and the store inventory, listed along with the prices.

\paragraph{}You will also see an (incomplete) list of available commands. Note that
many of the commands which work in the dungeon work in the stores as well,
but some do not, especially those which involve "using" objects.

\paragraph{}Stores do not always have everything in stock. As the game progresses,
they may get new items so check from time to time. Stores restock after
10000 game turns have passed, but the inventory will never change while
you are in town, even if you save the game and return. You must be in the
dungeon for the store to restock. Also, if you sell them an item, it may
get sold to a customer while you are adventuring, so don't always expect
to be able to get back everything you have sold. If you have
a lot of spare gold, you can purchase every item in a store, which will
induce the store owner to bring out new stock, and perhaps even retire.

\paragraph{}Store owners will not buy known harmful or useless items. If
an object is unidentified, they will pay you some base price for it.
Once they have bought it they will immediately identify the object. If
it is a good object, they will add it to their inventory. If it was a
bad bargain, they simply throw the item away. You can use this feature
to learn item flavors.
\paragraph{}
\begin{description}
\item[The General Store ("1")]
     The General Store sells foods, some clothing, torches, scrolls
     of phase door, scrolls of word of recall, oil, shovels, picks,
     and spikes. All of these items and some others can be sold
     back to the General store for money. The general store restocks
     like every store, but the inventory types never change.

\item[The Armoury ("2")]
     The Armoury is where the town's armour is fashioned. All sorts of
     protective gear may be bought and sold here. The deeper into the
     dungeon you progress the more exotic the equipment you will find stocked
     in the armoury. However, some armour types will never appear here
     unless you sell them.

\item[The Weaponsmith's Shop ("3")]
     The Weaponsmith's Shop is where the town's weapons are fashioned. Hand
     and missile weapons may be purchased and sold here, along with arrows,
     bolts, and shots. As with the armoury, not all weapon types will be
     stocked here, unless they are sold to the shop by the player first.

\item[The Temple ("4")]
     The Temple deals in healing potions, as well as bless scrolls, remove
     curse scrolls, and some approved priestly weapons, as well as prayer
     books.

\item[The Alchemy shop ("5")]
     The Alchemy Shop deals in all types of potions and scrolls, including
     stat restoring potions.

\item[The Magic User's Shop ("6")]
     The Magic User's Shop deals in all sorts of rings, wands, amulets, and
     staves, as well as magic user books.

\item[The Black Market ("7")]
     The Black Market will sell and buy anything at extortionate prices.
     However it occasionally has VERY good items in it. With the exception
     of artifacts, every item found in the dungeon may appear in the black
     market.

\item[Your Home ("8")]
     This is your house where you can store objects that you
     cannot carry on your travels, or will need at a later date.
\end{description}

\subsection{Within The Dungeon}
\paragraph{}Once your character is adequately supplied with food, light, armor, and
weapons, he is ready to enter the dungeon. Move on top of the `$>$' symbol
and use the "Down" command (``$>$'').

\paragraph{}Your character will enter a maze of interconnecting staircases and finally
arrive somewhere on the first level of the dungeon. Each level of the
dungeon is fifty feet high (thus dungeon level "Lev 1" is often called
"50 ft"), and is divided into (large) rectangular regions (several times
larger than the screen) by titanium walls. Once you leave a level by a
staircase, you will never again find your way back to that region of that
level, but there are an infinite number of other regions at that same "depth"
that you can explore later. Monsters, of course, can use the stairs,
and you may eventually encounter them again, but they will not chase you up
or down stairs.

\paragraph{}In the dungeon, there are many things to find, but your
character must survive many horrible and challenging encounters to find
the treasure lying about and take it safely back to the town to sell.

\paragraph{}There are two sources for light once inside the dungeon.
Permanent light which has been magically placed within rooms, and a
light source carried by the player. If neither is present, the character
will be unable to see.  This will affect searching, picking locks,
disarming traps, reading scrolls, casting spells, browsing books, etc.
So be very careful not to run out of light!

\paragraph{}A character must wield a torch or lamp in order to supply
his own light. A torch or lamp burns fuel as it is used, and once it is
out of fuel, it stops supplying light. You will be warned as the light
approaches this point.  You may use the "Fuel" command ("F") to refuel
your lantern (with flasks of oil) or your torch (with other torches), so
it is a good idea to carry extra torches or flasks of oil, as
appropriate. There are rumours of objects of exceptional power which
glow with their own never-ending light.

\subsection{Objects Found In The Dungeon}
\paragraph{}The mines are full of objects just waiting to be picked up and used. How
did they get there? Well, the main source for useful items are all the
foolish adventurers that proceeded into the dungeon before you. They get
killed, and the helpful creatures scatter the various treasure throughout
the dungeon. There are a few cursed items present in the dungeon -
either left behind as the result of a failed attempt to create a magic items,
or deliberately placed by evil sorcerers. Some will inconvenience you but
a very few are deadly.

\paragraph{}Only one object may occupy a given floor location, which may or may not also
contain one creature. Doors, rubble, traps, and staircases are considered
"objects" for this purpose. As below, any item may actually be a "pile" of
up to 99 identical items. With the right choice of "options", you may be
able to "stack" several items in the same grid.

\paragraph{}You pick up objects by moving on top of them. You can carry up to 23
different items in your backpack while wearing and wielding up to 12 others.
Although you are limited to 23 different items, each item may actually be a
"pile" of up to 99 similar items. If you "t"ake off an item, it will go
into your backpack if there is room: if there is no room in your backpack,
it will drop onto the floor, so be careful when swapping one wielded weapon
or worn piece of armor for another when your pack is full.

\paragraph{}You are, however, limited in the total amount of weight that you can carry.
If you exceed this value, you become slower, making it easier for monsters
to chase you. Note that there is no upper bound on how much you can carry,
if you do not mind being slow. Your weight "limit" is determined by your
strength.

\paragraph{}Many objects found within the dungeon have special commands for their use.
Wands must be Aimed, staves must be Used, scrolls must be Read, and potions
must be Quaffed. You may, in general, not only use items in your pack, but
also items on the ground, if you are standing on top of them. At the
beginning of the game all items are assigned a random 'flavor'. For example
potions of 'cure light wounds' could be 'red potions'. If you have never
used, sold, or bought one of these potions, you will only see the flavor.
You can learn what type of item it is by selling it to a store, using it and
noticing an effect (for example, quaffing a healing potion when injured), or
casting the spell identify. Lastly, items in stores that you have not
yet identified the flavor of will be labeled {unknown}

\paragraph{}Chests are complex objects, containing traps, locks, and possibly treasure
or other objects inside them once they are opened. Many of the commands
that apply to traps or doors also apply to chests and, like traps and doors,
these commands do not work if you are carrying the chest.

\paragraph{}One item in particular will be discussed here. The scroll of "Word of
Recall" can be found within the dungeon, or bought at the temple in town.
All classes start with one of these scrolls in their inventory.
It acts in two manners, depending upon your current location. If read
within the dungeon, it will teleport you back to town. If read in town, it
will teleport you back down to the deepest level of the dungeon which your
character has previously been on. This makes the scroll very useful for
getting back to the deeper levels of Angband. Once the scroll has been
read it takes a while for the spell to act, so don't expect it to save you
in a crisis. During this time the word 'recall' will appear on the bottom
of the screen below the dungeon. Reading a second scroll before the first
takes effect will cancel the action.

\paragraph{}You may "inscribe" any object with a textual inscription of your choice.
These inscriptions are not limited in length, though you may not be able to
see the whole inscription on the item. The game applies special meaning to
inscriptions containing any text of the form "\@\#" or "\@x\#" or "!x" or "!*",
see "command.txt".

\paragraph{}The game provides some "fake" inscriptions to help you keep track of your
possessions. Wands and staves which are known to be empty will be inscribed
with {empty}. Objects which have been tried at least once but haven't been
identified yet will be inscribed with "tried". Cursed objects are inscribed
with {cursed}. In addition while carrying or wielding weapons or armor you
may learn qualities of these items. {average} means that the item has no
magical bonuses. {magical} means that it has magical bonuses although these
bonuses may be negative. An item with negative bonuses is not necessarily
cursed. An {ego} item has special abilities. If the abilities are not
obvious to the wielder the item will get an {excellent} tag. If the item has
obvious abilities, such as an increase to a statistic it will be labeled
{splendid}. Lastly, if at any point you pick up an Artifact you will learn
its name immediately and it will be labeled {special}. However, you may not
be aware of all its powers. Note that these inscriptions are fake, and cannot
be removed.

\paragraph{}Only weapons and armor will receive these pseudo-identifications. Wands,
staves, rods, scrolls, potions and mushrooms can only get the {tried} label.
However, if using or consuming the item creates an obvious effect, you will
learn the flavor. For example, if you are injured and drink an unknown

\paragraph{}It is rumored that rings of power and extra rare spell books
may be found deeper in the dungeon\ldots

And lastly, a final warning: not all objects are what they seem. The line
between tasty food and a poisonous mushroom is a fine one, and sometimes a
chest full of treasure will grow teeth in its lid and bite your hand
off\ldots


\subsection{Cursed Objects}
\paragraph{}Some objects, mainly armor and weapons, have had curses laid upon them.
These horrible objects will look like any other normal item, but will
detract from your character's stats or abilities if worn. They will also
be impossible to remove until a remove curse is performed. In fact some
are so badly cursed that even this will not work, and more potent methods
are needed.

\paragraph{}If you wear or wield a cursed item, you will immediately
feel deathly cold, and the item will be marked with the fake inscription
"cursed".

\paragraph{}Shopkeepers will refuse to buy any item which is known to be
cursed, either by having been identified or by being marked with the
fake inscription.

\paragraph{}Weapons and armor can be uncursed by either spells of remove
curse or by attempting to enchant them with spells or scrolls.

\subsection{Mining}
\paragraph{}Some treasure within the dungeon can be found only by mining
it out of the walls. Many rich strikes exist within each level, but must
be found and mined. Quartz veins are the richest, yielding the most
metals and gems, but magma veins will have some hoards hidden within.

\paragraph{}Mining is rather difficult without a pick or shovel. Picks
and shovels have an additional magical ability expressed as `(+\#)'. The
higher the number, the better the magical digging ability of the tool. A
pick or shovel also has plusses to hit and damage, and can be used as a
weapon, because, in fact, it is one.

\paragraph{}When a vein of quartz or magma is located, the character may
wield his pick or shovel and begin digging out a section. When that
section is removed, he can locate another section of the vein and begin
the process again. Since granite rock is much harder to dig through, it
is much faster to follow the vein exactly and dig around the granite.
Eventually, it becomes easier to simply kill monsters and discover items
in the dungeon to sell, than to walk around digging for treasure. But,
early on, mineral veins can be a wonderful source of easy treasure.

\paragraph{}If the character has a scroll, staff, or spell of treasure
location, he can immediately locate all strikes of treasure within a
vein shown on the screen. This makes mining much easier and more
profitable.

\paragraph{}Note that a character with high strength and/or a heavy
weapon does not need a shovel/pick to dig, but even the strongest
character will benefit from a pick if trying to dig through a granite
wall.

\paragraph{}It is sometimes possible to get a character trapped within
the dungeon by using various magical spells and items. So it can be a
good idea to always carry some kind of digging tool, even when you are
not planning on tunneling for treasure.

\paragraph{}There are rumors of certain incredibly profitable rooms
buried deep in the dungeon and completely surrounded by titanium and
granite walls, requiring a digging implement or magical means to enter.
The same rumors imply that these rooms are guarded by incredibly
powerful monsters, so beware!

\subsection{Traps}
\paragraph{}There are many traps located in the dungeon of varying
danger. These traps are hidden from sight and are triggered only when
your character walks over them. If you have found a trap you can attempt
to "D"isarm it, but failure may mean activating it.

\paragraph{}There are some magical means to detecting all traps within a
certain radius.  If you cast one of these spells, there will be a
'Dtrap' green label on the bottom of the screen, below the dungeon map.
At some point in the dungeon you may see a line of green squares on the
floor. This line represents the extent of your detection. Beyond the
green line you are no longer in the safe region.

\paragraph{}Some monsters have the ability to create new traps on the
level that may be hidden, even if the player is in a detected zone. The
detection only finds the traps that exist at the time of detection, it
does not inform you of new ones that have since been created.

\subsection{Staircases, Secret Doors, Passages, and Rooms}
\paragraph{}Staircases are the manner in which you get deeper or climb
out of the dungeon. The symbols for the up and down staircases are the
same as the commands to use them. A ``$<$'' represents an up staircase
and a ``$>$'' represents a down staircase. You must move your character
over the staircase before you can use it.

\paragraph{}Each level has at least one up staircase and at least two
down staircases.  There are no exceptions to this rule. You may have
trouble finding some well hidden secret doors, or you may have to dig
through obstructions to get to them, but you can always find the stairs
if you look hard enough. Stairs, like titanium walls, and shop
entrances, cannot be destroyed by any means.

\paragraph{}Many secret doors are used within the dungeon to confuse and
demoralize adventurers foolish enough to enter. But with some luck, and
lots of concentration, you can find these secret doors. Secret doors
will sometimes hide rooms or corridors, or even entire sections of that
level of the dungeon. Sometimes they simply hide small empty closets or
even dead ends.  Secret doors always look like granite walls, just like
traps always look like normal floors.

\paragraph{}Creatures in the dungeon will generally know and use these
secret doors, and can often be counted on to leave them open behind them
when they pass through.

\paragraph{}For historical reasons, secret doors are never locked.

\subsection{Winning The Game}
\paragraph{}If your character has killed Sauron (a difficult task), who
lives on level 99 (4950') in the dungeon, a magical staircase will
appear that will allow you to finally reach level 100. Morgoth lurks on
this level of his dungeon, and you will not be able to go below his
level until you have killed him.  Try to avoid wandering around on level
100 unless you are ready for him, since he has a habit of coming at you
across the dungeon, the Mighty Hammer `Grond' in hand, to slay you for
your impudence.

\paragraph{}If you should actually survive the attempt of killing
Morgoth, you will receive the status of WINNER. You may continue to
explore, and may even save the game and play more later, but since you
have defeated the toughest creature alive, there is really not much
point. Unless you wish to listen to the rumors of a powerful ring buried
somewhere in the dungeon, or a suit of dragon scale mail that resists
everything\ldots

\paragraph{}When you are ready to retire, simply "commit suicide" (using
the "Q" key) to have your character entered into the high score list as
a winner. Note that until you retire, you can still be killed, so you
may want to retire before wandering into yet another horde of greater
demons.

\subsection{Upon Death and Dying}
\paragraph{}If your character falls below 0 hit points, he has died and
cannot be restored. A tombstone showing information about your character
will be displayed. You are also permitted to get a record of your
character, and all your equipment (identified) either on the screen or
in a file.

\paragraph{}Your character will leave behind a reduced save file, which
contains only the monster memory and your option choices. It may be
restored, in which case a new character is generated exactly as if the
file was not there, but the new player will find his monster memory
containing all the experience of past incarnations.

\paragraph{}There are a variety of ways to "cheat" death (including
using a special "cheating option") when it would otherwise occur. This
will fully heal your character, returning him to the town, and marking
him in various ways as a character which has cheated death. Cheating
death, like using any of the "cheating options", will prevent your
character from appearing on the high score list.


