\section{Attacking --- attack.txt}

\subsection{Attacking and Being Attacked}
\paragraph{}
Attacking is simple in Angband. If you move into a creature, you attack it.
You can attack from a distance by firing a missile or by magical means (such
as aiming a wand). Creatures attack in the same way. If they move into you,
they attack you. Some creatures can also cast spells from a distance, and
others can use various breath weapons (such as fire) on you from a distance.

\paragraph{}
Creatures in walls can not be attacked by wands or other magic attacks
normally stopped by walls, nor can they be shot at with bows and arrows.
Tunnelling into the wall (using the "tunnel" or "alter" command) will
allow you to attack any creature in the wall with your main weapon.
This applies to creatures which "pass through" walls: if they "bore
through" walls, the wall is no longer there, and the creature can be
targetted normally.

\paragraph{}
If you are wielding a weapon, the damage for the weapon is used when you hit
a creature. Otherwise you get a single punch which does minimal damage.

\paragraph{}
You may "w"ield one weapon for melee combat, and also one missile launcher
(bow, crossbow or sling). You may also wear one amulet (around the one and
only neck of the character), two rings (on the two "ring" fingers, i.e. the
third finger of each hand: a magic ring does not function when worn on any
other finger, nor may two be worn on the same finger), one light source,
and a full set of armor - body armor, shield, helmet, gloves, boots and
a cloak. Any or all of these items may provide powers to the character
in terms of bonuses to-hit, to-damage, to-armor class, or to other stats.

\paragraph{}
Firing a missile (while wielding the appropriate launcher) is the only way
to get the "full" power out of the missile. You may of course throw an
arrow at a monster without shooting it, but you will find the effects will
not be what you had hoped. 

\paragraph{}
Hits and misses are determined by ability to hit versus armor class. A hit
is a strike that does some damage; a miss may in fact reach a target, but
fails to do any damage. Higher armor classes make it harder to do damage,
and so lead to more misses. Characters with higher armor classes also
receive a damage reduction. This is not true for monsters, whose AC only
affects the character's difficulty to hit them.

\paragraph{}
If you wish to see how much damage your weapon will do, you can "I"nspect it.
You will find the number of blows and how much damage you would do per round,
including information on whether your weapon damages other types of
monsters differently.

\subsection{Monster Memories}
\paragraph{}
There are hundreds of different creatures in the pits of Angband, many of
which have the same letter symbol and color on the screen. The exact species
of a creature can be discovered by "l"ooking at it. It is also very difficult
to keep track of the capabilities of various creatures. Luckily, Angband
automatically keeps track of your experiences with a particular creature.
This feature is called the monster memory. Your monster memory recalls the
particular attacks of each creature (whether or not technically a monster)
which you have suffered, as well as recalling if you have observed them to
multiply or move erratically, or drop treasure, etc. Otherwise you would
simply have to take notes, which is an unnecessary bother.

\paragraph{}
If you have killed enough of a particular creature, or suffered enough
attacks, recalling the monster memory may also provide you with
information not otherwise available, such as a armor class, hit dice,
spell types, frequency of spell casting, or the amount of damage for
breaths or spells. These attacks will be color coded to inform you of
whether or not you currently resist a specific attack. Red or orange
means you do not resist it, yellow means you partially resist it, and
green means you resist it or are immune. If you attack a monster with
specific elemental attacks you will learn if the monster resists that
element or if they are immune. There are other magical means to learn
about monster's abilities that don't require you to actually experience
the attacks.
 
\paragraph{}
This memory can be passed on to a new character even after you die by means
of a re-used save file. (You must start the new character by opening the
existing save file of the dead character for this to happen: using the
"start new character" option will not give access to any monster memory.)

\subsection{Your Weapon}
\paragraph{}
Carrying a weapon in your backpack does you no good. You must wield a
weapon before it can be used in a fight. A secondary weapon can be kept
by keeping it in the backpack, and switching it with the primary weapon
when needed. This is most often used when switching between a melee
weapon and a digging tool, or when carrying two weapons, each of which
provides a rare power that the character needs at two separate times.

\paragraph{}
Weapons have two main magical characteristics, their enchanted ability
to hit and their enchanted ability to do damage, expressed as '(+\#,+\#)'.
A normal weapon would be '(+0,+0)'. Many weapons in Angband have
bonuses to hit and/or do damage. A very few weapons are cursed, and
have penalties that diminish your hit and/or damage bonuses. Cursed
weapons cannot be unwielded until the curse is lifted. Identifying a
weapon will inform you of the magical bonuses and penalties and whether
or not it is cursed.
  
\paragraph{}
Angband assumes that your youth in the rough environment near the
dungeons has taught you the relative merits of different weapons, and
displays as part of their description the damage dice which define their
capabilities. Any damage enchantment is added to the dice roll for that
weapon. The dice used for a given weapon is displayed as "XdY". The
number "X" indicates how many dice to roll, and number "Y" indicates how
many sides they have. A "2d6" weapon will thus give damage from 2 to 12,
plus any damage bonus. The weight of a weapon is also a consideration.
Heavy weapons may hit harder, but they are also harder to use.
Depending on your strength, dexterity, character class, and weapon
weight, you may get attack more quickly: high dexterity and strength and
low weapon weight are the main factors. Warriors may get up to a maximum
of 6 attacks per round: mages and priests are limited to only 4: other
classes may get up to 5. Your attacks per round with a weapon are
displayed as a decimal, e.g. 2.3 or 3.4 etc. The fractions take the form
of unused energy which is carried over to your next turn.

\paragraph{}
Missile weapons, such as bows, have their characteristics added to those
of the missile used, if the proper weapon/missile combination is used,
and then the launcher multiplier is applied to the total damage, making
missile weapons very powerful given the proper missiles, especially if
they are enchanted. Like weapons, "I"nspecting ammunition will tell you
how much damage you will do with your current missile launcher.

\paragraph{}
Although you receive any magical bonuses an unidentified weapon may possess
when you wield it, those bonuses will not be immediately added to the
displayed values of to-hit and to-dam on your character sheet. In order to
learn these plusses you must use the weapon in combat or use magical means of
identification. The same applies for missiles and launchers.

\paragraph{}
Finally, some rare weapons have special abilities. These are called ego
weapons, and are feared by great and meek. An ego weapon must be wielded
to receive the benefit of its abilities. It should be noted that some of
these items are considerably more powerful than others, and generally the
most powerful items are the rarest. Also important is the fact that not all
ego-items are nice. Some items will have an obvious effect, like an increase
in infravision, or extra strength. These effects will be noticed as soon as
you wield the item. Other effects, like most resistances, will need to be
learned. You can learn them by either suffering an appropriate attack, or
by using magical means of identification.

\paragraph{}
Some of the more common ego weapons are described at the end of this file.

\subsection{Your Armor Class}
\paragraph{}
Your armor class (or AC) is a number that describes the amount and the
quality of armor being worn. Armor class will generally run from about 0 to
200, though exceptionally good armor can improve even on the latter figure.

\paragraph{}
The larger your armor class, the more protective it is. A negative armor
class would actually help get you hit. Armor protects you in three manners.
First, it makes you harder to be hit for damage. A hit for no damage
counts as a miss, and is described as a miss. Secondly, good armor will
absorb some of the damage that your character would have taken from normal
attacks. Thirdly, acid damage is reduced by wearing body armor (but the
armor may be damaged instead). It is obvious that a high armor class is
vital for surviving the deeper levels of Angband.

\paragraph{}
Each piece of armour has a base armor value, which, like the damage from
weapons, is assumed known by the player, and a magic bonus, which will not
be displayed unless the armor has been identified or was bought in a store.

\paragraph{}
Armor class values are always displayed between a set of square brackets,
as "[\#]" or "[\#,+\#]". The first value is the base armor class of the armor.
The second number is the magical bonus of the item, which is only displayed
if known, and will always have a sign preceding the value. These plusses can
be determined by wielding the armor in combat and being hit. Note that a few
rings, amulets, and weapons also have the "[+\#]" notation, indicating that
they provide an armor bonus. Many pieces of heavy body armor will also have
a "(-\#)" (in normal brackets) before the "[\#,+\#]", which indicates that the
weight of the armor decreases your chances of hitting monsters. This can
range from nonexistent for very light armor to (-8) for the heaviest armor!

\subsection{Non-melee attacks and resistances}
\paragraph{}
The player may at some time gain access to non-melee attacks, and many
monsters also have them. Perhaps the most famous of this type of attack is
dragon breath, but monsters may also cast spells at the player, and vice
versa. This damage generally is not affected by armor class, and does not
need a hit roll to hit the player or monster being aimed at.

\paragraph{}
Some attacks are purely magical: attack spells which blind, confuse, slow,
scare or paralyze the target. In some cases, a melee attack may also do
this. These attacks are resisted by monsters of higher level (native to
deeper dungeon depths) and characters with a high saving throw - saving
throws being dependent on class, level and wisdom. There are also available
resistances to fear, blindness and confusion, and the power of "free
action" prevents magical paralysis and most slowing attacks (the player may
still be paralyzed by being "knocked out" in melee or by a stunning attack,
but this is very rare.) It should also be noticed that unique monsters
automatically pass their saving throws, and some monsters are naturally
resistant to confusion, fear and sleep-monster attacks. Some monsters may
have spells that 'cause wounds' that can be deadly if successful but do no
damage if the saving throw is passed.

\paragraph{}
Some melee attacks by monsters may drain a stat, as can some traps: this is
prevented by having that stat sustained. Drained stats are temporary and can
be restored on gaining a new character level or consuming rare items found
in the dungeon.

\paragraph{}
Some monsters may cast spells that teleport the player character. There is
no saving throw, except to those that would actually teleport him
up or down one dungeon level. Having resistance to nexus will also prevent
being level-teleported, but will not help against normal teleportation spell
attacks. The player may teleport monsters in the same way, with a spell,
wand or rod. No monsters, even Morgoth himself, resist this teleportation.
Yet...

\paragraph{}
Other attacks are usually element-based, including the aforementioned
example of dragon breath. Many monsters can breathe various attacks or
cast bolt or ball spells, and the player may also have access to bolt and
ball spells (or breathe like a dragon, in some rare circumstances). The
player, and the monsters, may be resistant to these forms of attack:
resistance is handled in different ways for the player and the monster,
and for different attack forms.

\paragraph{}
Bolt spells will hit the first monster (or the player) in the line of
fire: ball spells and breaths may centre on a target which may be hiding
behind other targets. Ball spells and breath weapons affect an area:
other monsters caught in the blast take reduced damage depending on how
close to the centre of the blast they are. Breath weapons are
proportional to a fraction of the monster's current hit points, with a
maximum cap on the damage (which is higher for the most common of such
attacks, owing to the fact that the resistances are also easier to
find). Bolt and ball spell damage is calculated differently - often (but
not always) relative to character or monster level.

\paragraph{}
In the case of fire, cold, lightning, acid and poison, if the monster
has resistance to a player attack of this kind it will take almost no
damage. If the player has one or more permanent sources of resistance,
he will take 1/3 of the damage he would normally take: if the player has
a temporary source of resistance (whether from potion, spell or item
activation), this will also reduce the damage to 1/3 of its normal
level, allowing the character to take only 1/9 damage if he has both
permanent and temporary resistance. Having more than one source of
permanent resistance confers no extra bonus, and using more than one
source of temporary resistance increases only the duration of the
resistance: in both cases, either the resistance is present or it is
not. But one permanent resistance and one temporary resistance are both
effective simultaneously.

\paragraph{}
Elemental attacks also have a chance to damage wielded equipment or
destroy items in the character's inventory. Fire attacks destroy
scrolls, staves, magic books and arrows. Acid attacks destroy scrolls,
staves, arrows, bolts and can damage armor. Electricity attacks can
destroy wands, rods, rings and amulets. Cold attacks can destroy
potions. Items in your inventory get a saving throw, and they are
unharmed if they pass it. Having resistance to the element will make an
item less likely to be destroyed. Items on the floor that get caught in
an elemental ball or breath are automatically destroyed without a saving
throw. Weapons, armor and chests can also be destroyed if they are
lying on the floor, but cannot be harmed if they are in your pack.

\paragraph{}
The character may also gain immunity to fire, cold, lightning and acid if
he is fortunate to find any of the few artifacts that provide these
immunities: immunity means that no damage is taken, and the character's
equipment is also totally protected. Immunities are EXTREMELY rare.

\paragraph{}
Another attack that the player will come into contact with all too often
is the soul-chilling nature of the undead, which can drain the character's
life experience. Some monsters have a life-draining melee attack, others may
cast ball or bolt spells or, in extreme cases, breathe the very force of the
netherworld (shortened by the game to "nether".) There are two powers which
are of assistance in this case: that of "hold life" will prevent 90\% of all
experience drains, and in the other 10\% of cases, the amount of experience
lost will be reduced by 90\%. That of "resistance to nether forces" will
provide resistance to nether bolts, balls and breaths, reducing the damage
and preventing any experience drains from these attacks, but has no effect
on melee "hits to drain experience". Monsters caught in the blast from a
nether ball or breath will take damage proportional to distance from the
centre of the attack, except for undead who are totally immune. The player
may find wands or rods of Drain Life, which similarly are ineffective on
those undead creatures which have no life to drain: however, the real
player equivalent attack spell is the priest/paladin spell of "Orb of
Draining", a ball spell which does damage to all monsters, double damage to
evil monsters, and is resisted by none.

\paragraph{}
Other attack forms are rarer, but may include: disenchantment (both in melee
or by a monster breath), chaos (breath or melee, which if unresisted
will cause the player to hallucinate and be confused, and may drain life
experience), nexus (which may teleport the player to the monster, away from
the monster, up or down a level, or swap over two of the player's "internal"
stats), light and darkness (which will blind a non-resistant character),
sound (which will stun a non-resistant character and may destroy his
potions), crystal shards (which will cut a non-resistant character and may
destroy his potions), inertia (which will slow a character even if he has
free action), gravity (which will blink a character, also stunning and
slowing), force (which will stun the character), plasma (which will stun,
and may destroy items which are vulnerable to either fire or lightning),
time (which may drain experience regardless of hold life, or drain stats
regardless of sustains), water bolts and balls (which may confuse and
stun, and do considerable damage from high-level monsters), ice bolts
(which may cut and stun, and damage potions), and mana bolts and balls
(the latter usually known as Mana Storms.) Magic missiles are included in
the "mana" category, whether cast by the monster or the player.

\paragraph{}
Some attacks may stun or cut the player. These can either be spells or
breath attacks (sound, water balls) or from melee. A stunned character
receives a penalty to hit and is much more likely to fail a spell or
activation. If a character gets very stunned, they may be knocked out
and at the mercy of the enemies. A cut character will slowly lose life
until healed either by potions, spells or natural regeneration. Both
stunning and cut status are displayed at the bottom of the screen.

\paragraph{}
There are resistances available to chaos, disenchantment, confusion, nexus,
sound, shards, light and darkness: all of these will reduce the damage and
prevent side-effects other than physical damage. With these resistances, as
with nether resistance, damage is a random fraction: for light and dark, it
is between 4/7 and 4/12, for sound and confusion it is between 5/7 and 5/12,
and for chaos, disenchantment, nexus, shards and nether it is between 6/7
and 6/12.

\paragraph{}
It should be noted that not all of these are actually vital to completing
the game: indeed, of the above list, only fire, cold, acid, lightning, poison
and confusion resists are regarded as truly vital, with blindness, chaos and
nether the next most desirable. Some attack forms are not resistable, but
thankfully these are rare: resist shards will prevent all other magical
attacks which cut (namely ice bolts), and resist sound will prevent all
magical stunning (force breath, water bolts and balls, ice bolts, gravity and
plasma), and confusion resistance will prevent confusion by a water bolt or
ball, but there is no resistance to the physical damage caused by these
following attacks: inertia, force, gravity, plasma, time, ice, water, mana.
There is no resistance to any of the side-effects of a time attack, or indeed
to anything but the stunning effects of a gravity attack.

\subsection{A note on speed}
\paragraph{}
Monsters which do not move at normal speed generally move "slowly" (-10 to
speed), "fast" (+10), "very fast" (+20) or "incredibly fast" (+30). (It will
surprise nobody that Morgoth\index{Morgoth} is one of the few monsters in
the last category.) This is further adjusted by the fact that any
non-unique monster may have a random adjustment from (-2) to (+2) to its
own speed. 

\paragraph{}
Generally, (+10) is exactly double normal speed, and (-10) exactly half.
(+20) is about three times normal speed, but after that there is less
noticeable improvement as speed goes higher - for instance, (+30) is not
quite four times normal speed, and higher values than this are largely
irrelevant. The player may find items which can be worn or wielded that
provide speed bonuses: these may include boots of speed, rings of speed
and a few very rare artifacts. Boots will provide a random 1d10 to speed:
rings of speed may be bigger than that - generally the best that the player
will get is two just over (+10), but individual rings of up to (+23) speed
have been known (although the record cursed one is much higher, at (-56):
only one unsuccessful roll away from being a (+56) ring...)

\paragraph{}
Separate from the question of permanent speed (as determined by the player's
speed items and the monster's natural speed) is that of temporary speed. The
player may cast a spell of haste-self, or use a potion, staff or rod of speed
or use an artifact activation to speed him temporarily: or a monster may cast
a haste-self spell, or be affected by another monster "shrieking for help"
or the player reading a scroll of aggravate monster. In all cases, (+10)
speed is added temporarily to the affected monster or player. Using two or
more sources of temporary speed is cumulative only in duration - one cannot
get from normal speed to (+20) using a potion and a spell of speed. Spells
of temporary slowing (including monsters breathing inertia or gravity) are
handled the same way, with exactly (-10) being subtracted from the player
or monster's speed temporarily, for the duration of the spell or breath's
effect.

\subsection{Ego weapons and armor}
Some of the ego weapons that you might find in the dungeon are listed below.
This will give you a small taste of the items that can be found. However if
you wish to discover these items on your own, you may not wish to continue.
Ego weapons are denoted by the following "names":

\subsubsection{Ego melee weapons}
\begin{description}
\item[(Defender)]
     A magical weapon that actually helps the wielder defend himself, thus
     increasing his/her armor class, and protecting him/her against damage
     from fire, cold, acid, lightning, and falls. This weapon also will
     increase your stealth, let you see invisible creatures, protect you
     from paralyzation and some slowing attacks, and help you regenerate hit
     points and mana faster. As a result of the regeneration ability, you
     will use up food somewhat faster than normal while wielding such a
     weapon. These powerful weapons also will sustain one stat, though this
     stat will vary from weapon to weapon.

\item[(Holy Avenger)]
     A Holy Avenger is one of the more powerful of weapons. A Holy Avenger
     will increase your wisdom and your armour class. This weapon will do
     extra damage when used against evil, demonic and undead creatures, and
     will also give you the ability to see invisible creatures. These
     weapons are basically extremely powerful versions of Blessed Blades
     and can be wielded by priests with no penalty. These weapons, like
     (Defender) weapons, also will sustain one random stat.

\item[(Blessed)]
     A blessed blade will increase your wisdom. If you are a priest,
     wielding a non-blessed sword or polearm causes a small penalty while
     attacking and may infuriate your god, decreasing the chances that he
     will accept your prayers: a blessed blade may be wielded without
     this penalty. Blessed blades also have one extra, random, power.

\item[Weapon of Westernesse]
     A Weapon of Westernesse is one of the more powerful weapons. It does
     extra damage against orcs, trolls, and giants, while increasing your
     strength, dexterity, and constitution. It also lets you see invisible
     creatures and protects from paralyzation and some slowing attacks.
     These blades were made by the Dunedain.

\item[Weapon of Extra Attacks]
     A weapon of extra attacks will allow the wielder to deliver extra
     attacks during each round.

\item[Elemental Branded Weapons]
     Each of the five elemental attacks has a corresponding weapon which
     will do treble its base damage to creatures not resistant to that
     element. (It should be noted that the magical damage bonus is not
     affected by this: a weapon of Flame (2d6) (+5,+6) does 6d6+6 damage
     per hit, not 6d6+18, against creatures which are not fire-resistant.)
     There are weapons of Flame, Frost, Lightning, Acid and Poison brands.

\item[Weapons of Slaying enemies]
     These weapons do extra damage against creatures of a vulnerable type.
     Weapons of Slay Evil and Slay Animal do double the base damage, while
     weapons of Slay Orc, Troll, Giant, Dragon, Demon and Undead do triple
     the base damage. As with elemental branded weapons, the magical damage
     bonus is not affected.

\item[Weapons of *Slaying* enemies]
     These weapons, in addition to doing extra damage to your enemies, have
     extra powers as well. In each case, one stat is increased. Weapons of
     *Slay* Dragon, Demon or Undead are also more powerful against their
     opponents, doing five times their base damage rather than the normal
     three.

\item[Weapon of Morgul]
     These blades are so foully cursed with evil, it is rumored that it is
     nearly impossible to remove them without a special curse-removing spell,
     much stronger than normal Remove Curse.

\item[Shovels and Picks of Digging]
     These powerful diggers will dig through granite as if it were mere wood,
     and mineral veins as if they were butter. Permanent rock is still an
     impassable obstacle.
\end{description}

\subsubsection{Ego Missile Launchers and Ammo:}
\begin{description}
\item[Launchers of Accuracy]
    These launchers have an unnaturally high to-hit number, making them
    extremely accurate.

\item[Launchers of Velocity]
    These launchers do an unnaturally high amount of damage due to their
    high to-dam number.

\item[Launchers of Extra Shots]
    These launchers allow the wielder to shoot more times per round than
    normal.

\item[Launchers of Extra Might]
    These launchers have a higher base damage than normally made launchers
    of their type. For instance, a Long Bow of Extra Might (x3)(+X,+Y)(+1)
    is really a Long Bow (x4)(+X,+Y) where (+X,+Y) is the standard to-hit
    and to-dam. As the damage multiplier with the bow affects *everything*
    - the base arrow damage, the magical damage bonus on both the bow and
    the arrow, and any bonuses for slaying or elemental-branded arrows -
    this makes it a powerful weapon.

\item[Ammo of Wounding]
    This ammunition - whether it be pebbles, iron shots, arrows, bolts,
    seeker arrows or seeker bolts - has big bonuses to-hit and to-damage.

\item[Ammo of Elemental Brands, and Ammo of Slaying enemies]
    This works in the same way as melee weapons of the same type: double
    damage for slay evil and slay animal, triple damage for all other
    slays and for all elemental brands. Unlike melee weapons, the slays
    and elemental brands *do* affect the magical damage bonus for ammo.

\item[Ammo of Backbiting]
    This cursed ammunition will never hit what it is aimed at. Rumour has
    it that it will curve round and hit the person who shot it in the
    back: but so far, this rumour has proved unfounded, and is believed
    to be false.
\end{description}

\paragraph{}
These are the most common types of ego-weapon: note that they are not the
ONLY ego-items available in the dungeon, there may be more.

\paragraph{}
Apart from these there are some very rare and well made weapons in the
dungeon with not necessarily any special abilities. These include Blades
of Chaos, Maces of Disruption, and Scythes of Slicing. They can also be
ego weapons like the ones above. For example, a Blade of Chaos (Holy
Avenger) is much more powerful than many artifact weapons!

\paragraph{}
Some pieces of armor will possess special abilities denoted by the following
names:

\subsubsection{Ego Armors and Shields:}
\begin{description}
\item[of Resist Acid, Lightning, Fire or Cold]
     A character wearing armor or a shield with one such resistance will
     take only 1/3 of normal damage from attacks involving the relevant
     element of acid, lightning, fire or cold. Note that multiple
     permanent sources of resistance are NOT cumulative: wearing two is
     no better than wearing one. However, armor which provides resistance
     to acid cannot itself be damaged by acid, and this is a good reason
     to wear more than one such piece of armor.
 
\item[of Resistance]
     A character wearing armor with this ability will have resistance to
     Acid, Cold, Fire, and Lightning as explained in each part above.

\item[Armor of Elvenkind]
     This is the same as Resistance armor, only generally better enchanted.
     It will make you more stealthy. This armor also possesses an extra
     resistance, at random from the following list: poison, light, dark,
     blindness, confusion, nexus, fear, nether, chaos, disenchantment,
     sound and shards.
 
\item[Robes of Permanence]
     These robes are designed especially for wizards. Just like Elvenkind
     armor, they provide resistance to fire, cold, acid, and electricity and
     cannot be damaged by acid. They sustain all of your stats and protect
     you from a good deal of all experience draining. Also like Elvenkind
     armor, they have one random resistance.

\item[Dragon Scale Mails]
     These extremely rare pieces of armour come in many different colors,
     each protecting you against the relevant dragons. Naturally they are
     all resistant to acid damage. They also occasionally allow you to
     breathe as a dragon would!
\end{description}

\subsubsection{Ego Helms:}
\begin{description}
\item[Stat Boosting Helms]
     There are magical helms found in the dungeon that have the ability
     to boost the wearer's intelligence, wisdom, or charisma. Helms of
     Beauty are the ones that boost charisma. In addition to boosting
     the relevant stat these helms will also prevent that stat from being
     drained.

\item[Crown of the Magi]
     This is the great crown of the wizards. The wearer will have an
     increased (and sustained) intelligence, and will also be given
     resistance against fire, frost, acid, and lightning. These valuable
     helms also have an additional random power.

\item[Crown of Might]
     This is the crown of the warriors. The wearer will have an
     increased and sustained strength, dexterity, and constitution,
     and will also be immune to any foe's attempt to slow or paralyze
     him or her.

\item[Crown of Lordliness]
     This is the great crown of the priests. The wearer will have an
     increased and sustained wisdom and charisma.

\item[Helm/Crown of Seeing]
     This is the great helmet or crown of the rogues. The wearer will be
     able to see invisible creatures, and will have an increased ability to
     locate traps and secret doors. It is also rumored that the wearer of
     such a helm will not be able to be blinded.

\item[Helm of Infravision]
     This helmet allows the character to see monsters even in total
     darkness, with the ability to see heat. Note that spellbooks are the
     same temperature as the surroundings, and so cannot be read unless
     some real light is present. (Some monsters which are invisible to
     normal vision can be seen under infravision.)

\item[Helm of Light]
     In addition to providing a permanent light source for the wearer, this
     helm also provides resistance against light-based attacks.

\item[Helm/Crown of Telepathy]
     This helm or crown grants the wearer the power of telepathy.

\item[Helm of Regeneration]
     This helm will help you regenerate hit points and mana more quickly
     than normal, allowing you to fight longer before needing to rest. You
     will use food faster than normal while wearing this helm because of
     the regenerative effects.

\item[Helms of Teleportation]
     This cursed helm will always keep the player on the hop, never sure
     where he is and never able to stay in one place.

\item[Crowns of Sickliness]
     Ruling is such a drain on one's health... This crown will debilitate
     its wearer, reducing his strength, dexterity and constitution.

\item[Helms/Crowns of Dullness]
     These helms can reduce your wisdom, intelligence and charisma.
\end{description}

\subsubsection{Ego Cloaks:}
\begin{description}
\item[Cloak of Protection]
     This finely made cloak will come with an unnaturally high enchantment
     and is not affected by elemental based attacks.

\item[Cloak of Stealth]
     This cloak will increase the wearer's stealth, making the wearer less
     likely to wake up sleeping monsters.

\item[Cloak of Aman]
     These exceptionally rare cloaks provide great stealth, have a very
     high enchantment, and one random resistance.

\item[Cloak of Enveloping]
     This cloak gets in the way of everything you try to do. You can't seem
     to hit monsters without having to shove aside several folds of cloth,
     and yet more layers cushion the blow and reduce the damage you do.

\item[Cloak of Irritation]
     This cloak not only envelops its wearer but aggravates all who see it.

\item[Cloaks of Vulnerability]
     This severely cursed cloak appears to be powerful, but in fact lets
     through the full force of every attack directed at it.
\end{description}

\subsubsection{Ego Gloves:}
\begin{description}
\item[Gloves of Free Action]
     The wearer of these gloves will find himself resistant to paralyzing
     attacks as well as some slowing attacks. Because of the special
     nature of these gloves, magic users may wear these gloves without
     incurring a mana penalty.

\item[Gloves of Slaying]
     These gloves will increase the wearer's fighting ability by boosting
     the wearer's to-hit and to-dam values.

\item[Gloves of Agility]
     These gloves will increase the wearer's dexterity. Because of the
     special nature of these gloves, magic users may wear these gloves
     without incurring a mana penalty.

\item[Gauntlets of Power]
     These spiked gauntlets will boost the wearer's strength as well as the
     wearer's to-hit and to-dam numbers.

\item[Gauntlets of Weakness]
     These gauntlets sap your strength.

\item[Gauntlets of Clumsiness]
     These gauntlets make you fumble.
\end{description}

\subsubsection{Ego Boots:}
\begin{description}
\item[Boots of Slow Descent] These boots protect the wearer from the
effects of small falls.

\item[Boots of Stealth] These boots increase the wearer's stealth, like
a Cloak of Stealth.

\item[Boots of Free Action] The wearer of these boots will find himself
resistant to paralyzing attacks as well as some slowing attacks.

\item[Boots of Speed] The wearer of these boots will become unnaturally
fast.

\item[Boots of Slowness] They seem to be stuck to the ground, and you
can hardly move at all while wearing these things.

\item[Boots of Annoyance] These boots will both slow the player and
aggravate monsters.
\end{description}

\paragraph{}
Once again, these are not necessarily the ONLY ego-items in the dungeon,
only the most common.

\paragraph{}
Apart from these there are some very rare and well-made armours in the
dungeon with not necessarily any special abilities. These include Shields of
Deflection, Adamantite Plate Mail, Mithril Plate Mail, Mithril Chain Mail,
and Elven Cloaks. The first four cannot be damaged by acid because of the
quality metals they contain.

\paragraph{}
There are rumors of unique "artifact" items in the dungeon - weapons and
armor of all types. Many of these are more powerful than even the greatest
ego-items: some are weak and have little more than a name to recommend them. 

