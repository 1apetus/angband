\section{Customization --- customize.txt}
\paragraph{}Angband supports multiple customizations. These can vary
from alternative methods for selecting objects, macros for different
commands, changing the visuals (using graphical tiles), creating
subwindows, and saving customizations for use on future characters.

\subsection{Special Keys}
\paragraph{}Certain special keys may be intercepted by the operating
system or the host machine, causing unexpected results. In general,
these special keys are control keys, and often, you can disable their
special effects.

\paragraph{}If you are playing on a UNIX or similar system, then Ctrl-C will
interrupt Angband. The second and third interrupt will induce a warning
bell, and the fourth will induce both a warning bell and a special message,
since the fifth will quit the game, after killing your character. Also,
Ctrl-Z will suspend the game, and return you to the original command shell,
until you resume the game with the "fg" command. There is now a compilation
option to force the game to prevent the "double ctrl-z escape death trick".
The Ctrl-\ and Ctrl-D and Ctrl-S keys should not be intercepted.
 
\paragraph{}It is often possible to specify "control-keys" without actually
pressing the control key, by typing a caret ("\^{}") followed by the key.
This is useful for specifying control-key commands which might be caught
by the operating system as explained above.

\paragraph{}Pressing backslash ("\textbackslash ") before a command will bypass all keymaps,
and the next keypress will be interpreted as an "underlying command" key,
unless it is a caret ("\^{}"), in which case the keypress after that will be
turned into a control-key and interpreted as a command in the underlying
angband keyset. The backslash key is useful for creating macro actions
which are not affected by any keymap definitions that may be in force, for
example, the sequence "\textbackslash " + "." + "6" will always mean "run east", even if
the "." key has been mapped to a different underlying command.

\paragraph{}The "0" and "\^{}" and "\textbackslash " keys all have special meaning when entered
at the command prompt, and there is no "useful" way to specify any of them
as an "underlying command", which is okay, since they would have no effect.

\paragraph{}For many input requests or queries, the special character ESCAPE
will abort the command. The "[y/n]" prompts may be answered with "y" or
"n", or escape. The "-more-" message prompts may be cleared (after reading
the displayed message) by pressing ESCAPE, SPACE, RETURN, LINEFEED, or by
any keypress, if the "quick\_messages" option is turned on.
 

\subsection{Command Counts}
\paragraph{}Some commands can be executed a fixed number of times by preceding
them with a count. Counted commands will execute until the count expires,
until you type any character, or until something significant happens, such
as being attacked. Thus, a counted command doesn't work to attack another
creature. While the command is being repeated, the number of times left
to be repeated will flash by on the line at the bottom of the screen.

\paragraph{}To give a count to a command, type 0, the repeat count, and then
the command. If you want to give a movement command and you are using the
original command set (where the movement commands are digits), press space
after the count and you will be prompted for the command.
 
\paragraph{}Counted commands are very useful for time consuming commands, as
they automatically terminate on success, or if you are attacked. You may
also terminate any counted command (or resting or running), by typing any
character. This character is ignored, but it is safest to use a SPACE or
ESCAPE which are always ignored as commands in case you type the command
just after the count expires.

\paragraph{}You can tell Angband to automatically use a repeat count of 99
with commands you normally want to repeat (open, disarm, tunnel, bash,
alter, etc) by setting the "always\_repeat" option.

\subsection{Squelch settings}
\paragraph{}Angband allows you to ignore specific items that you don't want
to see anymore. These items are 'squelched' and any similar items are
hidden from view. There are several ways to squelch an item. The easiest
way is to choose the 's' option when destroying an object. Whenever you
destroy an object you are forced to confirm the destruction. One of the
options is 's'. If you use this option, the object is dropped and then
hidden from view.

\paragraph{}Weapons and armor have quality squelch options. These allow you
to specify what types of weapons and armor you are no longer interested
in seeing. There is a quality setting for each weapon and armor type.
Squelching weapons and armor by destroying the object will prompt you with
a question about whether you wish to squelch all of that type of armor with
a certain quality setting. These quality settings are described below:
\begin{description} 
\item[bad] the weapon/armor has negative AC, to-hit
    or to-dam. It may or may not be cursed

\item[average] The weapon/armor has no pluses no minuses. It is
    non-magical.

\item[good] The weapon/armor has positive AC, to-hit or to-dam. However
    it does not have any special abilities, brands, slays, stat-boosts,
    resistances
 
\item[excellent, with no high resists] Weapons will have some slays.
    Armor will have resistances, but these are limited to rfire, rcold,
    racid and relec.

\item[excellent but not splendid] Weapons and armor may have high
    resists but they do not have any stat boosts or brands.

\item[Splendid] Squelching all splendid items only leaves artifacts
    unsquelched.  
\end{description}


\subsection{Using Inscriptions}
\paragraph{}You can inscribe items to change the selection process. Normally
when you select an item from your inventory you must enter a letter that
corresponds to the item. Since the order of your inventory changes as
items get added and removed, you may want to inscribe alternate methods of
item selection. Instead of entering letters, you can enter numbers, and
automatically assign certain items to specific numbers. The way to do this
is by using the inscribe "{}" command, which will add an inscription to the
item. You can add an inscription directly to an item or by using the "~"
memory list you can add an auto-inscription. Then every similar item you
find from then on will have that inscription placed on this item.

\paragraph{}If you enter a number between 0 and 9, the first item engraved
with "@\#" where "\#" is the number you entered will be selected. For example,
if you have a shovel engraved with "@0" and you type "w" (for wield) and
then 0, you will wield the shovel. This is very useful for macros (see
below), since you can use this to select an object regardless of its
location in your pack. For example, Angband automatically defines a macro
for the key "X" to do "w0". If you then engrave both your digging
instrument and your primary weapon with @0, pressing X will wield
whichever one is not being currently wielded (letting you quickly switch
between them). Multiple numbers can be engraved on the same object; for
example, if a sword is engraved with @1@0, then either "w1" or "w0" will
wield it. Since you may want different items defined with @0 you can also
make an inscription "@x\#" will act like "@\#" but only when the current
"Angband command" is "x". Thus you can put "@z4" on a rod and "@u4" on a
staff, and then use both "z4" and "u4" as desired.

\paragraph{}Perhaps the most commonly inscribed item are spellbooks. Casting
spells often takes at least 3 commands. The cast commmand (m or p) the
book selection and the spell selection. Then you may need to select a
target. For example, if you are a make and the beginner's spellbook is the
first in your inventory, casting maa will cast magic missile. However, if
you lose your spellbook, casting maa will cast the first spell in whatever
new book is in the top of your inventory. This can be a waste in the best
case scenario and exceedingly dangerous in the worst. To avoid this, it's
very useful to inscribe your spellbooks with "@m1, @m2, @m3" etc. Then if
you lose your first spellbook and attempt to cast magic missile by using
"m1a", you will not accidentally select the wrong spellbook, if the first
spellbook is not in your inventory.

\paragraph{}You can also use inscriptions to force the game to prompt you on
whether or not you want to use that object. Any object containing "!x" in
its inscription, where "x" is the current "angband command" (or containing
"!*" for all commands) will induce "verification" whenever that object is
"selected". Thus, inscribing, say, "!v!k!d" on an object will greatly reduce
the odds of you "losing" it by accident, and inscribing "!*" on an object will
allow you to be very paranoid about using the object. Note that "selling"
and "dropping" both use the "d" command.

\subsection{User Pref Files}
\paragraph{}Angband allows you to change various aspects of the game to suit
your tastes. You may define keymaps (changing the way Angband maps your
keypresses to underlying commands), create macros (allowing you to map a
single keypress to a series of keypresses), modify the visuals (allowing
you to change the appearance of monsters, objects, or terrain features),
change the colors (allowing you to make a given color brighter, darker,
or even completely different), or set options (turning them off or on).

\paragraph{}Angband stores your preferences in files called "user pref files",
which contain comments and "user pref commands", which are simple strings
describing one aspect of the system about which the user has a preference.
There are many ways to load a user pref file, and in fact, some of these
files are automatically loaded for you by the game. All of the files are
kept in the "lib/user/" directory, though you may have to use one of the
command line arguments to redirect this directory, especially on multiuser
systems. You may also enter single user pref commands directly, using the
special "Enter a user pref command" command, activated by "double quote".
You may have to use the "redraw" command (\^{}R) after changing certain of
the aspects of the game, to allow Angband to adapt to your changes.

\paragraph{}When the game starts up, after you have loaded an old character,
or created a new character, some user pref files are loaded automatically.
First, the "pref.prf" file is loaded. This file contains some user pref
commands which will work on all platforms. Then one of "font-xxx.prf"
(for normal usage) or "graf-xxx.prf" (for bitmap usage) is loaded. These
files contain attr/char changes to allow the monsters, objects, and/or
terrain features to look "better" on your system. Then the "pref-xxx.prf"
file is loaded. This file contains pre-defined system specific stuff
(macros, color definitions, etc). Then, the "user-xxx.prf" file is loaded.
This file contains user-defined system specific stuff. The "user-xxx.prf"
file is used as the "default" user pref file in many places. The "xxx" is
the "system suffix" for your system, taken from the "main-xxx.c" file which
was used to generate your executable. Finally, the "Race.prf", "Class.prf",
and "Name.prf" files are loaded, where "Race", "Class", and "Name" are
replaced by the actual race, class, and name of the current character.

\paragraph{}Several commands allow you to both load existing user pref files,
create new user pref files, append information to existing user pref files,
and/or interact with various of the user preferences in a more intuitive
way than the user pref commands allow. The commands include "Interact with
macros" (@), "Interact with visuals" (\%), and "Interact with colors" (\&),
described below.

\subsubsection{Macros}
\paragraph{}The "Interact with macros" command allows you to define or remove
"macros", which are mappings from a single logical keypress to a sequence
of keypresses, allowing you to use special keys on the keyboard, such as
function keys or keypad keys, possibly in conjunction with modifier keys,
to "automate" repetitive multi-keypress commands that you use a lot.

\paragraph{}Since macros represent keypress sequences, and not all
keypresses have a printable representation, macro triggers and actions
must often be "encoded" into a human readible form. This is done using
several types of encoding, including "\textbackslash xHH" for character
number HH in hexidecimal, "\textbackslash e" for the "escape" code,
"\textbackslash n" for the "newline" code, "\textbackslash r" for the
"return" code, "\textbackslash s" for the "space" code, "\textbackslash
\textbackslash " for backslash, ``\textbackslash \^{}'' for caret, and
"\^{}X" for the code for any "control" key "ctrl-X". Note that the
"action" of a macro will not be checked against other macro triggers
(unless the macro action contains a "control-backslash"), so you cannot
make infinite loops. You may specify extremely long macros, but you are
limited in length by the underlying input mechanisms, which in general
limit you to about 1024 keys in both triggers and actions.

\paragraph{}The special "\textbackslash " command (which must be encoded
in macros as "\textbackslash \textbackslash ") is very useful in macros, since it bypasses all
keymaps and allows the next keystroke to be considered a command in the
underlying Angband command set.  For a list of the Angband command set,
see the "command.txt" help file.  For example, a macro which maps
Shift-KP6 to "\textbackslash " + "." + "6" will induce the "run east"
behavior, regardless of what keyset the user has chosen, and regardless
of what keymaps have been defined.

\paragraph{}Macros can be specified in user pref files as a pair of
lines, one of the form "\verb+A:<str>+", which defines the encoded macro
action, and one of the form "\verb+P:<str>+", which defines the encoded macro
trigger.

\subsubsection{Keymaps}
\paragraph{}The "Interact with macros" command also allows you to define
"keymaps", which are vaguely related to macros. A keymap maps a single
keypress to a series of keypresses, which bypass both other keymaps and
any macros. Angband uses keymaps to map the original and the roguelike
keysets to the underlying command set, and allows the user to modify or
add keymaps of their own. Note that all keymap actions must be specified
using underlying commands, not keypresses from the original or roguelike
keysets. The original keyset is almost identical to the underlying
keyset, except that "numbers" are mapped to ";" plus a direction, "5" is
mapped to ",", and a few control-keys are mapped to various things. See
"command.txt" for the full set of underlying commands. Some uses for
keymaps include the ability to "disable" a command by mapping it to
"\textbackslash x00".

\paragraph{}Keymaps can be specified in user pref files as line of the
form "\verb+M:<T> <key> <str>+", where $<$T$>$ is the keyset (0/1 for
original/roguelike), <key> is the encoded trigger key, and $<$str$>$ is
the encoded keymap action.

\paragraph{}Note that any keys that are not required for access to the
underlying command set may be used by the user to extend the "keyset"
which is being used, by defining new "keymaps". To avoid the use of any
"keymaps", press backslash ("\textbackslash ") plus the "underlying
command" key. This is normally only used in "macro" definitions. You may
enter "control-keys" as a caret ("\^{}") plus the key (so "\^{}" + "p"
yields "\^{}P").


\subsubsection{Visuals}
\paragraph{}You can use the "Interact with visuals" command to change
various visual information, currently including the choice of what
attr/char values are used to represent various monsters, objects, or
terrain features. Note that in combination appropriate support in
"main-xxx.c", and with the use of the "use\_graphics" flag, you may be
able to specify that "graphic bitmaps" should be used instead of normal
"colored characters" for various things.

\paragraph{}When interactively modifying the attr/char values for
monsters, objects, or terrain features, pressing "n" or "N" will change
which entry you are changing, pressing "a" or "A" will rotate through
the available attr values, and pressing "c" or "C" will rotate though
the available char values. Note that attr/char values with the "high
bit" set may induce the display of special "graphic" pictures if the
"use\_graphics" flag is set, and your system supports the
"use\_graphics" flag.

\paragraph{}Note that this command can be abused in various ways, and if
you must do so, remember that you are only cheating yourself.

\paragraph{}Keymaps can be specified in user pref files as line of the
form "\verb+R:<N>:<A>/<C>+" or "\verb+K:<N>:<A>/<C>+" or
"\verb+F:<N>:<A>/<C>+" or "\verb+U:<N>:<A>/<C>+".

\subsubsection{Colors}
\paragraph{}The "Interact with colors" command allows you to change the
actual internal values used to display various colors. This command may
or may not have any effect on your machine. Advanced machines may allow
you to change the actual RGB values used to represent each of the 16
colors used by Angband, and perhaps even allow you to define new colors
which are not currently used by Angband.

\paragraph{}Colors can be specified in user pref files as line of the
form "\verb+V:<N>:<V>:<R>:<G>:<B>+".

\subsubsection{Options}
\paragraph{}The "Interact with options" command allows you to turn
options on or off. You may turn options off or on using the user pref
commands of the form "X:<option>" or "Y:<option>" respectively.


\subsection{Subwindows}
\paragraph{}In addition to the main window, you can create additional window
displays that have other secondary information on them. You can access the
subwindow menu by using "=" then "w". You can select what windows to
display. You may then need to make the window visible using the "window"
pull down menu from the menu bar. There are a variety of subwindow choices
and you should experiment to see which ones are the most useful for you.

