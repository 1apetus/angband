\section{Command --- command.txt}
\subsection{Command Descriptions} 
\paragraph{}The following command
descriptions are listed as the command name plus the default key to use
it. For those who prefer the original "roguelike" keyset, the name and
key of the roguelike command is also shown if it is different. Then
comes a brief description of the command, including information about
alternative methods of specifying the command in each keyset, when
needed.

\paragraph{}Some commands use the "repeat count" to automatically repeat
the command several times, while others use the "repeat count" to
specify a "quantity" for the command, and still others use it as an
"argument" of some kind. There is an option to "always\_repeat" certain
commands - please see the options help file.

\paragraph{}Most commands take no "energy" to perform, while other
commands only take energy when they cause the world to change in some
way. For example, attempting to read a scroll while blind does not use
any energy.

\subsubsection{Inventory Commands} \begin{description} \item[Inventory
        list (i)] Displays a list of objects being carried but not
        equipped. You can carry up to 23 different items, not counting
        those in your equipment. Often, many identical objects can be
        "stacked" into a "pile" which will count as a single item. This
        is always true of things like potions, scrolls, and food, but
        you may have to set options to allow wands, staves, and other
        such objects to stack. Each object has a weight, and if you
        carry more objects than your strength permits, you will begin to
        slow down. The amount of weight you can still carry without
        being overencumbered, or the amount of extra weight you are
        currently carrying is displayed at the top of the screen.
 
\item[Equipment list (e)] Use this command to display a list of the
    objects currently being used by your character. Your character has
    12 slots for equipment and 10 slots for ammunition (the quiver).
    Each equipment slot corresponds to a different location on the body,
    and each of which may contain only a single object at a time, and
    each of which may only contain objects of the proper "type", and
    which include WIELD (weapon), BOW (missile launcher), LEFT (ring),
    RIGHT (ring), NECK (amulet), LIGHT (light source), BODY (armor),
    OUTER (cloak), ARM (shield), HEAD (helmet), HANDS (gloves), FEET
    (boots). You must be wielding/wearing certain objects to take
    advantage of their special powers.

\item[Drop an item (d)] This drops an item from your inventory or
    equipment onto the dungeon floor. If the floor spot you are standing
    on already has an object in it, Angband will attempt to drop the
    item onto an adjacent space. Be warned that if the floor is full and
    you attempt to drop something, it may disappear and be destroyed.
    Doors and traps are considered objects for the purpose of
    determining if the space is occupied. This command may take a
    quantity, and takes some energy.

\item[Destroy an item (k) or Destroy an item (\^{}D)] This destroys an item
    in your inventory or on the dungeon floor.  If the selected pile
    contains multiple objects, you may specify a quantity. You must
    always (currently) verify this command. This command may take a
    quantity, and takes some energy.

\item[Wear/Wield equipment (w)] To wear or wield an object in your
    inventory, use this command. Since only one object can be in each
    slot at a time, if you wear or wield an item into a slot which is
    already occupied, the old item will be first be taken off, and may
    in fact be dropped if there is no room for it in your inventory.
    Wielding ammunition will add it to an empty slot in your quiver and
    prompt you to replace a type of ammunition if your quiver is already
    full.  This command takes some energy.

\item[Take off equipment (t) or Take off equipment (T)] Use this command
    to take off a piece of equipment and return it to your inventory.
    Occasionally, you will run into a cursed item which cannot be
    removed.  These items normally penalize you in some way and cannot
    be taken off until the curse is removed. If there is no room in your
    inventory for the item, your pack will overflow and you will drop
    the item after taking it off. You may also remove ammunition from
    your quiver with this command. This command takes some energy.
    \end{description}

\subsubsection{Movement Commands} 
\paragraph{}Moving (arrow keys, number
keys) or (arrow keys, number keys and 'yuhjklbn') This causes you to
move one step in a given direction. If the square you wish to move into
is occupied by a monster, you will attack it. If the square is occupied
by a door or a trap you may attempt to open or disarm it if the
appropriate option is set.  Preceeding this command with CTRL will cause
you to attack in the appropriate direction, but will not move your
character if no monster is there. These commands take some energy.
\begin{description} \item[Walk (with pickup) (;)] Moves one step in the
        given direction. The square you are moving into must not be
        blocked by walls or doors. You will pick up any items in the
        destination grid if the "always\_pickup" option is set.  You may
        also use the "original" direction keys (both keysets) or the
        "roguelike" direction keys (roguelike keyset) to walk in a
        direction. This command may take a count, requires a direction,
        and takes some energy.

    \item[Walk (W)] The walk command lets you willingly walk into a trap
        or a closed door, without trying to open or disarm it. This
        command may take a count, requires a direction, and takes some
        energy.

    \item[Run (.) or Run (,)] This command will move in the given
        direction, following any bends in the corridor, until you either
        have to make a "choice" between two directions or you are
        disturbed. You can configure what will disturb you by setting
        the disturbance options. You may also use shift plus the
        "roguelike" direction keys (roguelike keyset), or shift plus the
        "original" direction keys on the keypad (both keysets, some
        machines) to run in a direction. This command may take an
        argument, requires a direction, and takes some energy.

    \item[Go up staircase ($<$)] Climbs up an up staircase you are
        standing on. There is always at least one staircase going up on
        every level except for the town level (this doesn't mean it's
        easy to find). Going up a staircase will take you to a new
        dungeon level unless you are at 50 feet (dungeon level 1), in
        which case you will return to the town level. Note that whenever
        you leave a level (not the town), you will never find it again.
        This means that for all intents and purposes, any objects on
        that level are destroyed. This includes artifacts unless the
        "Create characters in preserve mode" option was set when your
        character was created, in which case the artifacts may show up
        again later. This command takes some energy.

    \item[Go down staircase (>)] Descends a down staircase you are
        standing on. There are always at least two staircases going down
        on each level, except for the town which has only one, and
        "quest" levels, which have none until the quest monster is
        killed. Going down a staircase will take you to a new dungeon
        level. See "Go Up Staircase" for more info.  This command takes
        some energy.  \end{description}

\subsubsection{Resting Commands} \begin{description} \item[Stay still
        (with pickup) (,) or Stay still (with pickup) (.)] Stays in the
        same square for one move. If you normally pick up objects you
        encounter, you will pick up whatever you are standing on. You
        may also use the "5" key (both keysets). This command may take a
        count, and takes some energy.

    \item[Get objects (g)] Pick up objects and gold on the floor beneath
        you. Picking up gold takes no time, and objects take 1/10th of a
        normal turn each (maximum time cost is a full turn). You may
        pick up objects until the floor is empty or your backpack is
        full.

    \item[Rest (R)] Resting is better for you than repeatedly staying
        still, and can be told to automatically stop after a certain
        amount of time, or when various conditions are met. In any case,
        you always wake up when anything disturbing happens, or when you
        press any key. To rest, enter the Rest command, followed by the
        number of turns you want to rest, or "*" to rest until your
        hitpoints and mana are restored, or "\&" to rest until you are
        fully "healed". This command may take an argument (used for the
        number of turns to rest), and takes some energy.
\end{description}

\subsubsection{Searching Commands} \begin{description} \item[Search (s)]
        This command can be used to locate hidden traps and secret doors
        in the spaces adjacent to the player. More than a single turn of
        searching will be required in most cases, so it is affected by
        the 'always repeat' option. You should always search a chest
        before trying to open it, since they are generally trapped. This
        command can take a count, which is useful if you are fairly sure
        of finding something eventually, since the command stops as soon
        as anything is found. This command takes some energy.

    \item[Toggle search mode (S) or Toggle search mode (\#)] This command
        will take you into and out of search mode. When first pressed,
        the message "Searching" will appear at the bottom of the screen.
        You are now taking two turns for each command, one for the
        command and one turn to search. This means that you are taking
        twice the time to move around the dungeon, and therefore twice
        the food. Search mode will automatically turn off if you are
        disturbed. You may also turn off search mode by entering the
        Search Mode command again.  \end{description}

\subsubsection{Alter Commands} \begin{description} \item[Tunnel (T) or
        Tunnel (\^{}T)] Tunnelling or mining is a very useful art. There
        are many kinds of rock, with varying hardness, including
        permanent rock (permanent), granite (very hard), quartz veins
        (hard), magma veins (soft), and rubble (very soft). Quartz and
        Magma veins may be displayed in a special way, and may sometimes
        contain treasure, in which case they will be displayed in a
        different way. Rubble sometimes covers an object. It is hard to
        tunnel unless you are wielding a heavy weapon or a shovel or a
        pick. Tunnelling ability increases with strength and weapon
        weight. This command may take a count, is affected by the
        "always\_repeat" option, requires a direction, and takes some
        energy.

    \item[Open a door or chest (o)] To open an object such as a door or
        chest, you must use this command. If the object is locked, you
        will attempt to pick the lock based on your disarming ability.
        If you open a trapped chest without disarming the traps first,
        the trap will be set off. Some doors will be jammed shut and may
        have to be forced open. Opening will automatically attempt to
        pick any lock doors. You may need several tries to open a door
        or chest. This command may take a count, is affected by the
        "always\_repeat" option, requires a direction, and takes some
        energy.

    \item[Close a door (c)] Non-intelligent and some other creatures
        cannot open doors, so shutting doors can be quite valuable.
        Furthermore, monsters cannot see you behind closed doors, so
        closing doors may allow you to buy some time without being
        attacked. Broken doors cannot be closed.  Bashing a door open
        may break it. This command may take a count, is affected by the
        "always\_repeat" option, requires a direction, and takes some
        energy.

    \item[Jam a door (j) or Spike a door (S)] Many monsters can simply
        open closed doors, and can eventually get through a locked door.
        You may therefore occasionally want to jam a door shut with iron
        spikes. Each spike used on the door will make it harder to bash
        down the door, up to a certain limit.  Smaller monsters are less
        able to bash down doors. In order to use this command, you must
        be carrying iron spikes. This command requires a direction, and
        takes some energy.

    \item[Bash a door (B) or Force a door (f)] This command allows you
        to bash down jammed doors. Your bashing ability increases with
        strength. Bashing open a door can (briefly) throw you off
        balance. Doors that are stuck, or which have been jammed closed
        with spikes can only be opened by bashing, and all closed doors
        can be bashed open if desired. Bashing a door open may
        permanently break it so that it can never be closed. This
        command may take a count, is affected by the "always\_repeat"
        option, requires a direction, and takes some energy.

    \item[Disarm a trap or chest (D)] You can attempt to disarm traps on
        the floor or on chests. If you fail, there is a chance that you
        will blunder and set it off. You can only disarm a trap after
        you have found it (usually with the Search command). This
        command may take a count, is affected by the "always\_repeat"
        option, requires a direction, and takes some energy.

    \item[Alter (+)] This special command allows the use of a single
        keypress to select any of the "obvious" commands above (attack,
        tunnel, bash, open, disarm, close), and, by using macros or
        keymaps, to combine this keypress with directions. In general,
        this allows the use of the "control" key plus the appropriate
        "direction" key (including the roguelike direction keys in
        roguelike mode) as a kind of generic "alter the terrain feature
        of an adjacent grid" command. This command may take a count, is
        affected by the "always\_repeat" option, requires a direction,
        and takes some energy.  \end{description}

\subsubsection{Spell and Prayer Commands} 
\paragraph{}Browse a book (b)
or Peruse a book (P) Only mages, rogues, and rangers can read magic
books, and only priests and paladins can read prayer books. Warriors
cannot read any books. When this command is used, all of the spells or
prayers contained in the selected book are displayed, along with
information such as their level, the amount of mana required to cast
them, and whether or not you know the spell or prayer.

\paragraph{}Gain new spells or prayers (G) Use this command to actually
learn new spells or prayers. When you are able to learn new spells or
prayers, the word "Study" will appear on the status line at the bottom
of the screen.  If you have a book in your possession, containing spells
or prayers which you may learn, then you may choose to study that book.
If you are a mage, rogue, or ranger, you may actually choose which spell
to study.  If you are a priest or paladin, your gods will choose a
prayer for you.  There are nine books of each type, five of which are
normally found only in the dungeon. This command takes some energy.

\paragraph{}Cast a spell (m) To cast a spell, you must have previously
learned the spell and must have in your inventory a book from which the
spell can be read. Each spell has a chance of failure which starts out
fairly large but decreases as you gain levels. If you don't have enough
mana to cast a spell, you will be prompted for confirmation. If you
decide to go ahead, the chance of failure is greatly increased, and you
may wind up paralyzed for several turns. Since you must read the spell
from a book, you cannot be blind or confused while casting, and there
must be some light present. This command takes some energy: the higher
your level, the less it takes, but the higher the spell level, the more
it takes.

\paragraph{}Pray a prayer (p) To pray effectively, you must have
previously learned the prayer and must have in your inventory a book
from which the prayer can be read. Each prayer has a chance of being
ignored which starts out fairly large but decreases as you gain levels.
If you don't have enough mana to cast a spell, you will be prompted for
confirmation. If you decide to go ahead, the chance of failure is
greatly increased, and you may lose a point of constitution.  Since you
must read the prayer from a book, you cannot be blind or confused while
praying, and there must be some light present.  This command takes some
energy - as with spells, higher caster level means less energy used, but
higher spell level means more.

\subsubsection{Object Manipulation Commands} 
\paragraph{}Eat some food
(E) You must eat regularly to prevent starvation. As you grow hungry, a
message will appear at the bottom of the screen saying "Hungry".  If you
go hungry long enough, you will become weak, then start fainting, and
eventually, you may will die of starvation. You may use this command to
eat food in your inventory. Note that you can sometimes find food in the
dungeon, but it is not always wise to eat strange food. This command
takes some energy.

\paragraph{}Fuel your lantern/torch (F) If you are using a torch and
have more torches in your pack, or you are using a lantern and have
flasks of oil in your pack, then your can "refuel" them with this
command. Torches and Lanterns are limited in their maximal fuel. In
general, two flasks will fully fuel a lantern and two torches will fully
fuel a torch. This command takes some energy.

\paragraph{}Quaff a potion (q) Use this command to drink a potion.
Potions affect the player in various ways, but the effects are not
always immediately obvious.  This command takes some energy.

\paragraph{}Read a scroll (r) Use this command to read a scroll. Scroll
spells usually have an area effect, except for a few cases where they
act on other objects.  Reading a scroll causes the parchment to
disintegrate as the scroll takes effect. Most scrolls which prompt for
more information can be aborted (by pressing escape), which will stop
reading the scroll before it disintegrates. This command takes some
energy.
 
\paragraph{}Inscribe an object (\{ )This command inscribes a string on an
object. The inscription is displayed inside curly braces after the
object description. The inscription is limited to the particular object
(or pile) and is not automatically transferred to all similar objects.
Under certain circumstances, Angband will display "fake" inscriptions on
certain objects ("cursed", "broken", "tried", "empty", "NN\% off") when
appropriate. These "fake" inscriptions remain all the time, even if the
player chooses to add a "real" inscription on top of it later.

\paragraph{}In addition, Angband will occasionally place a "real"
inscription on an object for you, normally as the result of your
character getting a "feeling" about the item. All characters will get
"feelings" about weapons and armor after carrying them for a while.
Warriors (who understand weapons and armor best) get the most detailed
feelings, and get them faster than other classes. Rogues (used to
handling all sorts of stolen goods) are also very good, as are paladins
(basically warriors who are religious.) Rangers are not so well-versed
in melee weapons, being normally people of the outdoors and the
wilderness who prefer archery and camouflage to heavy metal armor and
weaponry, while priests and mages simply are not experienced enough in
matters concerning melee to be able to tell much about an item - though
a priest will realise the extent of his knowledge fairly quickly, as
they are often called upon to fight for their faith and have learned a
little from these experiences and divine insight.  Mages, frankly,
haven't a clue. But then, they don't need it when they can learn the
spell of Identify very early on in the game.

\paragraph{}An item labeled as "{empty}" was found to be out of charges,
and an item labeled as "{tried}" is a "flavored" item which the
character has used, but whose effects are unknown. Certain inscriptions
have a meaning to the game, see "@\#", "@x\#", "!*", and "!x", in the
section on inventory object selection.

\paragraph{}Uninscribe an object (\}) This command removes the
inscription on an object. This command will have no effect on "fake"
inscriptions added by the game itself.

\subsubsection{Magical Object Commands} 
\paragraph{}Activate an artifact
(A) You have heard rumors of special weapons and armor deep in the Pits,
items that can let you breathe fire like a dragon or light rooms with
just a thought. Should you ever be lucky enough to find such an item,
this command will let you activate its special ability. Special
abilities can only be used if you are wearing or wielding the item. This
command takes some energy.

\paragraph{}Aim a wand (a) or Zap a wand (z) Wands must be aimed in a
direction to be used. Wands are magical devices, and therefore there is
a chance you will not be able to figure out how to use them if you
aren't good with magical devices. They will fire a shot that affects the
first object or creature encountered or fire a beam that affects
anything in a given direction, depending on the wand. An obstruction
such as a door or wall will generally stop the effects from traveling
any farther. This command requires a direction and can use a target.
This command takes some energy.

\paragraph{}Use a staff (u) or Zap a staff (Z) This command will use a
staff. A staff is normally very similar to a scroll, in that they
normally either have an area effect or affect a specific object. Staves
are magical devices, and there is a chance you will not be able to
figure out how to use them.  This command takes some energy.
 
\paragraph{}Zap a rod (z) or Activate a rod (a) Rods are extremely
powerful magical items, which cannot be burnt or shattered, and which
can have either staff-like or wand-like effects, but unlike staves and
wands, they don't have charges.  Instead, they draw on the ambient
magical energy to recharge themselves, and therefore can only be
activated once every few turns. The recharging time varies depending on
the type of rod.  This command may require a direction (depending on the
type of rod, and whether you are aware of its type) and can use a
target.  This command takes some energy.

\subsubsection{Throwing and Missile Weapons} 
\paragraph{}Fire an item
(f) or Fire an item (t) This command will fire a will allow you to fire
a missile from either your quiver or your inventory provided it is the
appropriate ammunition for the current BOW you have equipped.  You may
not fire an item without a BOW equipped. Fired ammunition has a chance
of breaking. This command takes some energy.

\paragraph{}Fire default ammo at nearest (h) or (TAB) If you have a
missile weapon equipped and the appropriate ammunition in your quiver,
you can use this command to fire at the nearest visible enemy. This
command will cancel itself if you lack a launcher, ammunition or a
visible target that is in range. The first ammunition of the correct
type found in the quiver is used.  This command takes some energy.

\paragraph{}Throw an item (v) You may throw any object carried by your
character.  Depending on the weight, it may travel across the room or
drop down beside you.  Only one object from a pile will be thrown at a
time. Note that throwing an object will often cause it to break, so be
careful!  If you throw something at a creature, your chances of hitting
it are determined by your pluses to hit, your ability at throwing, and
the object's pluses to hit. Once the creature is hit, the object may or
may not do any damage to it. You've heard rumors that some objects found
in the dungeon can do huge amounts of damage when thrown, but you're not
sure which objects those are.... Note that flasks of oil will do a
fairly large chunk of damage to a monster on impact, supposedly
representing fire damage, but it works against fire elementals too - not
that it's exactly worth it by that stage of dungeon exploration. If you
are wielding a missile launcher compatible with the object you are
throwing, then you automatically use the launcher to fire the missile
with much higher range, accuracy, and damage, than you would get by just
throwing the missile. Throw, like fire, requires a direction. Targeting
mode (see the next command) can be invoked with "*" at the "Direction?"
prompt. This command takes some energy.

\paragraph{}Targeting Mode (*) This will allow you to aim your spells
and such at a specific monster or grid, so that you can point directly
towards that monster or grid (even if this is not a "compass" direction)
when you are asked for a direction. You can set a target using this
command, or you can set a new target at the "Direction?" prompt when
appropriate. At the targeting prompt, you have many options. First of
all, targetting mode starts targetting nearby monsters which can be
reached by "projectable" spells and thrown objects. In this mode, you
can press "t" (or "5" or ".") to select the current monster, space to
advance to the next monster, "-" to back up to the previous monster,
direction keys to advance to a monster more or less in that direction,
"r" to "recall" the current monster, "q" to exit targetting mode, and
"p" (or "o") to stop targetting monsters and enter the mode for
targetting a location on the floor or in a wall. Note that if there are
no nearby monsters, you will automatically enter this mode.  Note that
hitting "o" is just like "p", except that the location cursor starts on
the last examined monster instead of on the player.  In this mode, you
use the "direction" keys to move around, and the "q" key to quit, and
the "t" (or "5" or ".") key to target the cursor location. Note that
targetting a location is slightly "dangerous", as the target is
maintained even if you are far away. To cancel an old target, simply hit
"*" and then ESCAPE (or "q"). Note that when you cast a spell or throw
an object at the target location, the path chosen is the "optimal" path
towards that location, which may or may not be the path you want.
Sometimes, by clever choice of a location on the floor for your target,
you may be able to convince a thrown object or cast spell to squeeze
through a hole or corridor that is blocking direct access to a different
grid. Launching a ball spell or breath weapon at a location in the
middle of a group of monsters can often improve the effects of that
attack, since ball attacks are not stopped by interposed monsters if the
ball is launched at a target.
        
\subsubsection{Looking Commands} 
\paragraph{}Full screen map (M) This
command will show a map of the entire dungeon, reduced by a factor of
nine, on the screen. Only the major dungeon features will be visible
because of the scale, so even some important objects may not show up on
the map. This is particularly useful in locating where the stairs are
relative to your current position, or for identifying unexplored areas
of the dungeon.

\paragraph{}Locate player on map (L) or Where is the player (W) This
command lets you scroll your map around, looking at all sectors of the
current dungeon level, until you press escape, at which point the map
will be re-centered on the player if necessary. To scroll the map
around, simply press any of the "direction" keys. The top line will
display the sector location, and the offset from your current sector.

\paragraph{}Look around (l) or Examine things (x) This command is used
to look around at nearby monsters (to determine their type and health)
and objects (to determine their type). It is also used to find out what
objects (if any) are under monsters, and if a monster is currently
inside a wall, and what is under the player.  When you are looking at
something, you may hit space for more details, or to advance to the next
interesting monster or object, or minus ("-") to go back to the previous
monster or object, or a direction key to advance to the nearest
interesting monster or object (if any) in that general direction, or "r"
to recall information about the current monster race, or "q" or escape
to stop looking around. You always start out looking at "yourself".

\paragraph{}Inspect an item (I) This command lets you inspect an item.
This will tell you things about the special powers of the object, as
well as attack information for weapons. It will also tell you what
resistances or abilities you have noticed for the item and if you have
not yet completely identified all properties.
        
\paragraph{}List visible monsters ([ <- not -> ]) This command lists all
monsters that are visible to you, telling you how many there are of each
kind. It also tells you whether they are asleep, and where they are
(relative to you).
% TODO work out these ['s

\paragraph{}List visible items (]) This command lists all items that are
visible to you, telling you how of each there are and where they are on
the level relative to your current location.

\subsubsection{Message Commands} 
\paragraph{}Repeat level feeling
(\^{}F) Repeats the feeling about the dungeon level that you got when
you first entered the level.

\paragraph{}View previous messages (\^{}P) This command shows you all
the recent messages. You can scroll through them, or exit with ESCAPE.

\paragraph{}Take notes (:) This command allows you to take notes, which
will then appear in your message list and your character history
(prefixed with "Note:").

\subsubsection{Game Status Commands} 
\paragraph{}Character Description
(C) Brings up a full description of your character, including your skill
levels, your current and potential stats, and various other information.
From this screen, you can change your name or use the file character
description command to save your character status to a file. That
command saves additional information, including your background, your
inventory, and the contents of your house.

\paragraph{}Check knowledge (~) This command allows you to ask about the
knowledge possessed by your character. Currently, this includes being
able to list all known "artifacts", "uniques", and "objects". Normally,
once an artifact is "generated", and "lost", it can never again be
found, and will become "known" to the player. With "preserve" mode, an
artifact can never be "lost" until it is "known" to the player.  In
either case, any "known" artifacts not in the possession of the player
will never again be "generated". The "uniques" are special "unique"
monsters which can only be killed once per game.  Certain "objects" come
in "flavors", and you must determine the effect of each "flavor" once
for each such type of object, and this command will allow you to display
all "known" flavors.  Inscribing items in this list will cause you to
similarly inscribe all similar items you find from this point on.

\paragraph{}Interact with the system (!) Allow the user to interact with
the underlying visual system.  This command is currently unused.
 
\subsubsection{Saving and Exiting Commands} 
\paragraph{}Save and Quit
(Ctrl-X) To save your game so that you can return to it later, use this
command. Save files will also be generated (hopefully) if the game
crashes due to a system error. After you die, you can use your savefile
to play again with the same options and such.

\paragraph{}Save (Ctrl-S) This command saves the game but doesn't exit
Angband. Use this frequently if you are paranoid about having your
computer crash (or your power go out) while you are playing.

\paragraph{}Quit (commit suicide) (Q) Kills your character and exits
Angband. You will be prompted to make sure you really want to do this,
and then asked to verify that choice. Note that dead characters are dead
forever.

\subsubsection{User pref file commands}

\paragraph{}Interact with options (=) Allow you to interact with
options. Note that using the "cheat" options may mark your savefile as
unsuitable for the high score list. You may change normal options using
the "X" and "Y" user pref commands. The "window" options allow you to
specify what should be drawn in any of the special sub-windows (not
available on all platforms). See the help file "options.txt" for more
info.  You can also interact with macros and keymaps under this menu.

\paragraph{}Interact with macros - option submenu Allow you to interact
with macros.  You may load or save macros from user pref files, create
macros of various types, or define keymaps. You must define a "current
action", shown at the bottom of the screen, before you attempt to use
any of the "create macro" commands, which use that "current action" as
their action. This is a horrible interface, and will be fixed
eventually.

\paragraph{}Interact with visuals - option submenu.  Allow you to
interact with visuals. You may load or save visuals from user pref
files, or modify the attr/char mappings for the monsters, objects, and
terrain features.  You must use the "redraw" command (\^{}R) to redraw
the map after changing attr/char mappings.  NOTE: It is generally easier
to modify visuals via the "knowledge" menus.

\paragraph{}Interact with colors - option submenu.  Allow the user to
interact with colors. This command only works on some systems.  NOTE: It
is commonly used to brighten the 'Light Dark' color (eg. Cave Spiders)
on displays with bad alpha settings.
 
\subsubsection{Help} 
\paragraph{}Help (?) Brings up the Angband on-line
help system. Note that the help files are just text files in a
particular format, and that other help files may be available on the
Net. In particular, there are a variety of spoiler files which do not
come with the standard distribution. Check the place you got Angband
from or ask on the newsgroup rec.games.roguelike.angband about them.

\paragraph{}Identify Symbol (/) Use this command to find out what a
character stands for. For instance, by pressing "/.", you can find out
that the "." symbol stands for a floor spot. When used with a symbol
that represents creatures, the this command will tell you only what
class of creature the symbol stands for, not give you specific
information about a creature you can see. To get that, use the Look
command.

\paragraph{}There are three special symbols you can use with the
Identify Symbol command to access specific parts of your monster memory.
Typing Ctrl-A when asked for a symbol will recall details about all
monsters, typing Ctrl-U will recall details about all unique monsters,
and typing Ctrl-N will recall details about all non-unique monsters.

\paragraph{}If the character stands for a creature, you are
asked if you want to recall details. If you answer yes,
information about the creatures you have encountered with that
symbol is shown in the Recall window if available, or on the
screen if not. You can also answer "k" to see the list sorted by
number of kills, or "p" to see the list sorted by dungeon level
the monster is normally found on. Pressing ESCAPE at any point
will exit this command.

\paragraph{}Game Version (V) This command will tell you what
version of Angband you are using.  For more information, see the
"version.txt" help file.

\paragraph{}Command lists (Enter) This brings up a little window in the
middle of the screen, in which you can select what command you would
like to use by browsing.  Useful for beginners.

\subsubsection{Extra Commands} 
\paragraph{}Toggle Choice Window (\^{}E)
Toggles the display in any sub-windows (if available) which are
displaying your inventory or equipment.

\paragraph{}Redraw Screen (\^{}R) This command adapts to various changes
in global options, and redraws all of the windows. It is normally only
necessary in abnormal situations, such as after changing the visual
attr/char mappings, or enabling "graphics" mode.

\paragraph{}Load screen dump (left-paren) This command loads a
"snap-shot" of the current screen from the file "dump.txt", and displays
it on the screen.

\paragraph{}Save screen dump (right-paren) This command dumps a
"snap-shot" of the current screen to the file "dump.txt", including
encoded color information. The command has three variants: - text, a
simple ascii dump of the screen concatenated with a dump of the color
attributes. It may be viewed in color with the '(' command.  - html,
suitable for viewing in a web browser.  - forum embedded html for
vBulletin, suitable for pasting in web forums like
\verb+http://angband.oook.cz/forums+.

