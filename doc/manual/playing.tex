\section{Playing the Game --- playing.txt}
\paragraph{}Most of your interaction with Angband will take the form of
``commands''.  Every Angband command consists of an ``underlying
command'' plus a variety of optional or required arguments, such as a
repeat count, a direction, or the index of an inventory object. Commands
are normally specified by typing a series of keypresses, from which the
underlying command is extracted, along with any encoded arguments. You
may choose how the standard ``keyboard keys'' are mapped to the
``underlying commands'' by choosing one of the two standard ``keysets'',
the ``original'' keyset or the ``roguelike'' keyset.

\paragraph{}The original keyset is very similar to the ``underlying''
command set, with a few additions (such as the ability to use the
numeric ``directions'' to ``walk'' or the ``5'' key to ``stay still'').
The roguelike keyset provides similar additions, and also allows the use
of the h/j/k/l/y/u/b/n keys to ``walk'' (or, in combination with the
shift or control keys, to run or alter), which thus requires a variety
of key mappings to allow access to the underlying commands used for
walking/running/altering. In particular, the ``roguelike'' keyset
includes many more ``capital'' and ``control'' keys, as shown below.

\paragraph{}Note that any keys that are not required for access to the
underlying command set may be used by the user to extend the ``keyset''
which is being used, by defining new ``keymaps''. To avoid the use of
any ``keymaps'', press backslash (``\textbackslash '') plus the
``underlying command''
key. This is normally only used in ``macro'' definitions. You may enter
``control-keys'' as a caret (``\^{}'') plus the key (so ``\^{}'' + ``p''
yields ``\^{}P'').

\paragraph{}Some commands allow an optional ``repeat count'', which
allows you to tell the game that you wish to do the command multiple
times, unless you press a key or are otherwise disturbed. To enter a
``repeat count'', type '0', followed by the numerical count, followed by
the command. You must type ``space'' before entering certain commands.
Skipping the numerical count yields a count of 99.  An option allows
certain commands (open, disarm, alter, etc) to auto-repeat.

\paragraph{}Some commands will prompt for extra information, such as a direction, an
inventory or equipment item, a spell, a textual inscription, the symbol of a
monster race, a sub-command, a verification, an amount of time, a quantity,
a file name, or various other things. Normally you can hit return to choose
the ``default'' response, or escape to cancel the command entirely.

\paragraph{}Some commands will prompt for a spell or an inventory item. Pressing
space (or '*') will give you a list of choices. Pressing ``-'' (minus) selects
the item on the floor. Pressing a lowercase letter selects the given item.
Pressing a capital letter selects the given item after verification. Pressing
a numeric digit `\#' selects the first item (if any) whose inscription contains
``@\#'' or ``@x\#'', where ``x'' is the current ``underlying command''. You may only
specify items which are ``legal'' for the command. Whenever an item inscription
contains ``!*'' or ``!x'' (with ``x'' as above) you must verify its selection.

\paragraph{}Some commands will prompt for a direction. You may enter a
``compass'' direction using any of the ``direction keys'' shown below.
Sometimes, you may specify that you wish to use the current ``target'',
by pressing ``t'' or ``5'', or that you wish to select a new target, by
pressing ``*'' (see ``Target'' below).

\paragraph{}
\begin{tabular}{rcl}
   Original & Keyset & Directions \\
\hline
7 & 8 & 9\\
4 & 5 & 6\\
1 & 2 & 3\\
\end{tabular}

\paragraph{}
\begin{tabular}{rcl}
    Roguelike & Keyset & Directions \\
    \hline
      y & k & u \\
      h &   & l \\
      b & j & n \\
\end{tabular}

\paragraph{}Each of the standard keysets provides some short-cuts over the ``underlying
commands''. For example, both keysets allow you to ``walk'' by simply pressing
an ``original'' direction key (or a ``roguelike'' direction key if you are using
the roguelike keyset), instead of using the ``walk'' command plus a direction.
The roguelike keyset allows you to ``run'' or ``alter'' by simply holding the
shift or control modifier key down while pressing a ``roguelike'' direction key,
instead of using the ``run'' or ``alter'' command plus a direction. Both keysets
allow the use of the ``5'' key to ``stand still'', which is most convenient when
using the original keyset.

\paragraph{}Note that on many systems, it is possible to define ``macros'' to various
keys, or key combinations, so that it is often possible to make macros which,
for example, allow the use of the shift and/or control modifier keys, plus a
numeric keypad key, to specify the ``run'' or ``alter'' command, with the given
direction, regardless of any keymap definitions, by using the fact that you
can always, for example, use ``\textbackslash '' + ``.'' + ``6'', to specify ``run
east''.

\subsection{Original Keyset Command Summary}
\begin{tabular}{cc|cc}
a & Aim a wand & A & Activate an artifact\\
b & Browse a book & B & Bash a door\\
c & Close a door & C & Character description\\
d & Drop an item & D & Disarm a trap\\
e & Equipment list & E & Eat some food\\
f & Fire an item & F & Fuel your lantern/torch\\
g & Get objects on floor & G & Gain new spells/prayers\\
h & (unused) & H & (unused)\\
i & Inventory list & I & Observe an item\\
j & Jam a door & J & (unused)\\
k & Destroy an item & K & (unused)\\
l & Look around & L & Locate player on map\\
m & Cast a spell & M & Full dungeon map\\
n & (unused) & N & (unused)\\
o & Open a door or chest & O & (unused)\\
p & Pray a prayer & P & (unused)\\
q & Quaff a potion & Q & Quit (commit suicide)\\
r & Read a scroll & R & Rest for a period\\
\end{tabular}
\paragraph{}
\begin{tabular}{cc|cc}
s & Search for traps/doors & S & Toggle search mode\\
t & Take off equipment & T & Dig a tunnel\\
u & Use a staff & U & (unused)\\
v & Throw an item & V & Version info\\
w & Wear/wield equipment & W & (unused)\\
x & (unused) & X & (unused)\\
y & (unused) & Y & (unused)\\
z & Zap a rod & Z & (unused)\\
! & Interact with system & \^{}A & (special - debug command)\\
@ & (unused) & \^{}B & (unused)\\
\# & (unused) & \^{}C & (special - break)\\
\$ & (unused) & \^{}D & (unused)\\
\% & (unused) & \^{}E & Toggle choice window\\
\^{} & (special - control key) & \^{}F & Repeat level feeling\\
\& & (unused) & \^{}G & (unused)\\
$\ast$ & Target monster or location & \^{}H & (unused)\\
( & Load screen dump & \^{}I & (special - tab)\\
) & Dump screen dump & \^{}J & (special - linefeed)\\
\{ & Inscribe an object & \^{}K & (unused)\\
\} & Uninscribe an object & \^{}L & (unused)\\
$[$ & Display visible monster list & \^{}M & (special - return)\\
$]$ & Display visible object list & \^{}N & (unused)\\
- & Walk (flip pickup) & \^{}O & (unused)\\
\_ & Enter store & \^{}P & Show previous messages\\
+ & Alter grid & \^{}Q & (unused)\\
= & Set options & \^{}R & Redraw the screen\\
; & Walk (with pickup) & \^{}S & Save and don't quit\\
: & Take notes & \^{}T & (unused)\\
' & (unused) & \^{}U & (unused)\\
" & Enter a user pref command & \^{}V & Repeat last command\\
, & Stay still (with pickup) & \^{}W & (special - wizard mode)\\
$<$ & Go up staircase & \^{}X & Save and quit\\
. & Run & \^{}Y & (unused)\\
$>$ & Go down staircase & \^{}Z & (special - borg command)\\
\textbackslash & (special - bypass keymap) & $|$ & (unused)\\
` & (special - escape) & $\sim$ & Check knowledge\\
/ & Identify symbol & ? & Help\\
\end{tabular}

\subsection{Roguelike Keyset Command Summary}
\begin{tabular}{cc|cc}
a & Zap a rod (Activate) & A & Activate an artifact\\
b & (walk - south west) & B & (run - south west)\\
c & Close a door & C & Character description\\
d & Drop an item & D & Disarm a trap or chest\\
e & Equipment list & E & Eat some food\\
f & Bash a door (force) & F & Fuel your lantern/torch\\
g & Get objects on floor & G & Gain new spells/prayers\\
h & (walk - west) & H & (run - west)\\
i & Inventory list & I & Observe an item\\
j & (walk - south) & J & (run - south)\\
\end{tabular}

\paragraph{}
\begin{tabular}{cc|cc}
k & (walk - north) & K & (run - north)\\
l & (walk - east) & L & (run - east)\\
m & Cast a spell & M & Full dungeon map\\
n & (walk - south east) & N & (run - south east)\\
o & Open a door or chest & O & (unused)\\
p & Pray a prayer & P & Browse a book\\
q & Quaff a potion & Q & Quit (commit suicide)\\
r & Read a scroll & R & Rest for a period\\
s & Search for traps/doors & S & Jam a door (Spike)\\
t & Fire an item & T & Take off equipment\\
u & (walk - north east) & U & (run - north east)\\
v & Throw an item & V & Version info\\
w & Wear/wield equipment & W & Locate player on map (Where)\\
x & Look around & X & (unused)\\
y & (walk - north west) & Y & (run - north west)\\
z & Aim a wand (Zap) & Z & Use a staff (Zap)\\
! & Interact with system & \^{}A & (special - debug command)\\
@ & (unused) & \^{}B & (alter - south west)\\
\# & Toggle search mode & \^{}C & (special - break)\\
\$ & (unused) & \^{}D & Destroy item\\
\% & (unused) & \^{}E & Toggle choice window\\
\^{} & (special - control key) & \^{}F & Repeat level feeling\\
\& & (unused) & \^{}G & (unused)\\
$\ast$ & Target monster or location & \^{}H & (alter - west)\\
( & Load screen dump & \^{}I & (special - tab)\\
) & Dump screen dump & \^{}J & (alter - south)\\
\{ & Inscribe an object & \^{}K & (alter - north)\\
\} & Uninscribe an object & \^{}L & (alter - east)\\
$[$ & Display visible monster list & \^{}M & (special - return)\\
$]$ & Display visible object list & \^{}N & (alter - south east)\\
- & Walk (flip pickup) & \^{}O & (unused)\\
\_ & Enter store & \^{}P & Show previous messages\\
+ & Alter grid & \^{}Q & (unused)\\
= & Set options & \^{}R & Redraw the screen\\
; & Walk (with pickup) & \^{}S & Save and don't quit\\
: & Take notes & \^{}T & Dig a Tunnel\\
' & (unused) & \^{}U & (alter - north east)\\
` &  Enter a user pref command & \^{}V & Repeat last command\\
, & Run & \^{}W & (special - wizard mode)\\
$<$ & Go up staircase & \^{}X & Save and quit\\
. & Stay still (with pickup) & \^{}Y & (alter - north west)\\
$>$ & Go down staircase & \^{}Z & (special - borg command)\\
\textbackslash & (special - bypass keymap) & $|$ & (unused)\\
` & (special - escape) & $\sim$ & Check knowledge\\
/ & Identify symbol & ? & Help\\
\end{tabular}

\subsection{Special Keys}
\paragraph{}Certain special keys may be intercepted by the operating system or
the host machine, causing unexpected results. In general, these special keys
are control keys, and often, you can disable their special effects.

\paragraph{}If you are playing on a UNIX or similar system, then Ctrl-C will
interrupt Angband. The second and third interrupt will induce a warning
bell, and the fourth will induce both a warning bell and a special message,
since the fifth will quit the game, after killing your character. Also,
Ctrl-Z will suspend the game, and return you to the original command shell,
until you resume the game with the ``fg'' command. There is now a compilation
option to force the game to prevent the ``double ctrl-z escape death
trick''.
The Ctrl-\ and Ctrl-D and Ctrl-S keys should not be intercepted.
 
\paragraph{}It is often possible to specify ``control-keys'' without actually
pressing the control key, by typing a caret (``\^{}'') followed by the key.
This is useful for specifying control-key commands which might be caught
by the operating system as explained above.

\paragraph{}Pressing backslash (``\textbackslash '') before a command will bypass all keymaps,
and the next keypress will be interpreted as an ``underlying command'' key,
unless it is a caret (``\^{}''), in which case the keypress after that will be
turned into a control-key and interpreted as a command in the underlying
angband keyset. The backslash key is useful for creating macro actions
which are not affected by any keymap definitions that may be in force, for
example, the sequence ``\textbackslash '' + ``.'' + ``6'' will always mean ``run east'', even if
the ``.'' key has been mapped to a different underlying command.

\paragraph{}The ``0'' and ``\^{}'' and ``\textbackslash '' keys all have special meaning when entered
at the command prompt, and there is no ''useful`` way to specify any of them
as an ''underlying command``, which is okay, since they would have no effect.

\paragraph{}For many input requests or queries, the special character ESCAPE
will abort the command. The ``[y/n]'' prompts may be answered with ``y'' or
``n'', or escape. The ``-more-'' message prompts may be cleared (after reading
the displayed message) by pressing ESCAPE, SPACE, RETURN, LINEFEED, or by
any keypress, if the ``quick\_messages'' option is turned on.
 

\subsection{Command Counts}
 
\paragraph{}Some commands can be executed a fixed number of times by preceding
them with a count. Counted commands will execute until the count expires,
until you type any character, or until something significant happens, such
as being attacked. Thus, a counted command doesn't work to attack another
creature. While the command is being repeated, the number of times left
to be repeated will flash by on the line at the bottom of the screen.

\paragraph{}To give a count to a command, type 0, the repeat count, and then
the command. If you want to give a movement command and you are using the
original command set (where the movement commands are digits), press space
after the count and you will be prompted for the command.
 
\paragraph{}Counted commands are very useful for time consuming commands, as
they automatically terminate on success, or if you are attacked. You may
also terminate any counted command (or resting or running), by typing any
character. This character is ignored, but it is safest to use a SPACE or
ESCAPE which are always ignored as commands in case you type the command
just after the count expires.

\paragraph{}You can tell Angband to automatically use a repeat count of 99
with commands you normally want to repeat (open, disarm, tunnel, bash,
alter, etc) by setting the ``always\_repeat'' option.
 
 
\subsection{Selection of Objects}
\paragraph{}Many commands will also prompt for a particular object to be used.
For example, the command to read a scroll will ask you which of the
scrolls that you are carrying that you wish to read. In such cases, the
selection is made by typing a letter of the alphabet. The prompt will
indicate the possible letters, and will also allow you to type the key
``*'', which causes all of the available options to be described. The list
of choices will also be shown in the Choice window, if you are using a
windows environment and windows are turned on. Often you will be able to
press ``/'' to select an object from your equipment instead of your
inventory. Pressing space once will have the same effect as ``*'', and
the second time will cancel the command and run the ``i'' or ``e'' command.
 
\paragraph{}The particular object may be selected by an upper case or a lower
case letter. If lower case is used, the selection takes place
immediately. If upper case is used, then the particular option is
described, and you are given the option of confirming or retracting that
choice. Upper case selection is thus safer, but requires an extra key
stroke. Also see the ``!*'' and ``!x'' inscriptions, below.

\paragraph{}For many commands, you can also use ``-'' to select an object on the
floor where you are standing. This lets you read scrolls or quaff
potions, for example, off the dungeon floor without picking them up.

\paragraph{}If you enter a number between 0 and 9, the first item engraved
with ``@\#'' where ``\#'' is the number you entered will be selected. For example,
if you have a shovel engraved with ``@0'' and you type ``w'' (for wield) and
then 0, you will wield the shovel. This is very useful for macros (see
below), since you can use this to select an object regardless of its
location in your pack. For example, Angband automatically defines a macro
for the key ``X'' to do ``w0''. If you then engrave both your digging
instrument and your primary weapon with @0, pressing X will wield
whichever one is not being currently wielded (letting you quickly switch
between them). Multiple numbers can be engraved on the same object; for
example, if a sword is engraved with @1@0, then either ``w1'' or ``w0'' will
wield it. Normally, you inscribe ``@1@0'' on your primary weapon, and
``@2@0'' on your digger or secondary weapon. Note that an inscription
containing ``@x\#'' will act like ``@\#'' but only when the current ``Angband
command'' is ``x''. Thus you can put ``@z4'' on a rod and ``@u4'' on a staff,
and then use both ``z4'' and ``u4'' as desired.

\paragraph{}Note that any object containing ``!x'' in its inscription, where
``x'' is the current ``angband command'' (or containing ``!*'' ever) will induce
``verification'' whenever that object is ``selected''. Thus, inscribing, say,
``!f!k!d'' on an object will greatly reduce the odds of you ``losing'' it by
accident, and inscribing ``!*'' on an object will allow you to be very paranoid
about the object. Note that ``selling'' and ``dropping'' both use the
``d'' command.

\subsection{User Pref Files}
\paragraph{}Angband allows you to change various aspects of the game to suit
your tastes. You may define keymaps (changing the way Angband maps your
keypresses to underlying commands), create macros (allowing you to map a
single keypress to a series of keypresses), modify the visuals (allowing
you to change the appearance of monsters, objects, or terrain features),
change the colors (allowing you to make a given color brighter, darker,
or even completely different), or set options (turning them off or on).

\paragraph{}Angband stores your preferences in files called ``user pref
files'', which contain comments and ``user pref commands'', which are
simple strings describing one aspect of the system about which the user
has a preference.  There are many ways to load a user pref file, and in
fact, some of these files are automatically loaded for you by the game.
All of the files are kept in the ``lib/user/'' directory, though you may
have to use one of the command line arguments to redirect this
directory, especially on multiuser systems. You may also enter single
user pref commands directly, using the special ``Enter a user pref
command'' command, activated by ``double quote''.  You may have to use
the ``redraw'' command (\^{}R) after changing certain of the aspects of
the game, to allow Angband to adapt to your changes.

\paragraph{}When the game starts up, after you have loaded an old character,
or created a new character, some user pref files are loaded automatically.
First, the ``pref.prf'' file is loaded. This file contains some user pref
commands which will work on all platforms. Then one of ``font-xxx.prf''
(for normal usage) or ``graf-xxx.prf'' (for bitmap usage) is loaded. These
files contain attr/char changes to allow the monsters, objects, and/or
terrain features to look ``better'' on your system. Then the
``pref-xxx.prf''
file is loaded. This file contains pre-defined system specific stuff
(macros, color definitions, etc). Then, the ``user-xxx.prf'' file is loaded.
This file contains user-defined system specific stuff. The
``user-xxx.prf''
file is used as the ``default'' user pref file in many places. The
``xxx'' is
the ``system suffix'' for your system, taken from the ``main-xxx.c'' file which
was used to generate your executable. Finally, the ``Race.prf'',
``Class.prf'',
and ``Name.prf'' files are loaded, where ``Race'', ``Class'', and
``Name'' are
replaced by the actual race, class, and name of the current character.

\paragraph{}Several commands allow you to both load existing user pref
files, create new user pref files, append information to existing user
pref files, and/or interact with various of the user preferences in a
more intuitive way than the user pref commands allow. The commands
include ``Interact with macros'' (@), ``Interact with visuals'' (\%),
and ``Interact with colors'' (\&), described below.

\subsubsection{User Pref Files (Macros)}
\paragraph{}The ``Interact with macros'' command allows you to define or remove
``macros'', which are mappings from a single logical keypress to a sequence
of keypresses, allowing you to use special keys on the keyboard, such as
function keys or keypad keys, possibly in conjunction with modifier keys,
to ``automate'' repetitive multi-keypress commands that you use a lot.

\paragraph{}Since macros represent keypress sequences, and not all
keypresses have a printable representation, macro triggers and actions
must often be ``encoded'' into a human readible form. This is done using
several types of encoding, including ``\textbackslash xHH'' for
character number HH in hexidecimal, ``\textbackslash e'' for the
``escape'' code, ``\textbackslash n'' for the ``newline'' code,
``\textbackslash r'' for the ``return'' code, ``\textbackslash s'' for
the ``space'' code, ``\textbackslash\textbackslash '' for backslash,
``\textbackslash \^{}'' for caret, and ``\^{}X'' for the code for any
``control'' key ``ctrl-X''.  Note that the ``action'' of a macro will
not be checked against other macro triggers (unless the macro action
contains a ``control-backslash''), so you cannot make infinite loops.
You may specify extremely long macros, but you are limited in length by
the underlying input mechanisms, which in general limit you to about
1024 keys in both triggers and actions.

\paragraph{}The special ``\textbackslash'' command (which must be
encoded in macros as ``\textbackslash\textbackslash'') is very useful in
macros, since it bypasses all keymaps and allows the next keystroke to
be considered a command in the underlying Angband command set.  For a
list of the Angband command set, see the ``command.txt'' help file.  For
example, a macro which maps Shift-KP6 to ``\textbackslash '' + ``.'' +
``6'' will induce the ``run east'' behavior, regardless of what keyset
the user has chosen, and regardless of what keymaps have been defined.

\paragraph{}Macros can be specified in user pref files as a pair of
lines, one of the form ``A:$<$str$>$'', which defines the encoded macro
action, and one of the form ``P:$<$str$>$'', which defines the encoded macro
trigger.

\subsubsection{User Pref Files (Keymaps)} 
\paragraph{}The ``Interact with macros'' command also allows you to
define ``keymaps'', which are vaguely related to macros. A keymap maps a
single keypress to a series of keypresses, which bypass both other
keymaps and any macros. Angband uses keymaps to map the original and the
roguelike keysets to the underlying command set, and allows the user to
modify or add keymaps of their own. Note that all keymap actions must be
specified using underlying commands, not keypresses from the original or
roguelike keysets. The original keyset is almost identical to the
underlying keyset, except that ``numbers'' are mapped to ``;'' plus a
direction, ``5'' is mapped to ``,'', and a few control-keys are mapped
to various things. See ``command.txt'' for the full set of underlying
commands.  Some uses for keymaps include the ability to ``disable'' a
command by mapping it to ``\textbackslash x00'',

\paragraph{}Keymaps can be specified in user pref files as line of the
form ``M:$<$T$>$ $<$key$>$ $<$str$>$'', where $<$T$>$ is the keyset (0/1
for original/roguelike), $<$key$>$ is the encoded trigger key, and
$<$str$>$ is the encoded keymap action.

\subsubsection{User Pref Files (Visuals)}
\paragraph{}You can use the ``Interact with visuals'' command to change
various visual information, currently including the choice of what
attr/char values are used to represent various monsters, objects, or
terrain features. Note that in combination appropriate support in
``main-xxx.c'', and with the use of the ``use\_graphics'' flag, you may
be able to specify that ``graphic bitmaps'' should be used instead of
normal ``colored characters'' for various things.

\paragraph{}When interactively modifying the attr/char values for
monsters, objects, or terrain features, pressing ``n'' or ``N'' will
change which entry you are changing, pressing ``a'' or ``A'' will rotate
through the available attr values, and pressing ``c'' or ``C'' will rotate
though the available char values. Note that attr/char values with the
``high bit'' set may induce the display of special ``graphic'' pictures if
the ``use\_graphics'' flag is set, and your system supports the
``use\_graphics'' flag.

\paragraph{}Note that this command can be abused in various ways, and if
you must do so, remember that you are only cheating yourself.

\paragraph{}Keymaps can be specified in user pref files as line of the
form\\
``R:$<$N$>$:$<$A$>$/$<$C$>$'' or ``K:$<$N$>$:$<$A$>$/$<$C$>$'' or\\
``F:$<$N$>$:$<$A$>$/$<$C$>$'' or ``U:$<$N$>$:$<$A$>$/$<$C$>$''.

\subsubsection{User Pref Files (Colors)}
\paragraph{}The ``Interact with colors'' command allows you to change the actual
internal values used to display various colors. This command may or may
not have any effect on your machine. Advanced machines may allow you to
change the actual RGB values used to represent each of the 16 colors used
by Angband, and perhaps even allow you to define new colors which are not
currently used by Angband.

\paragraph{}Colors can be specified in user pref files as line of the
form\\
``V:$<$N$>$:$<$V$>$:$<$R$>$:$<$G$>$:$<$B$>$''.

\subsubsection{User Pref Files (Options)}
\paragraph{}The ``Interact with options'' command allows you to turn options
on or off. You may turn options off or on using the user pref commands
of the form ``X:$<$option$>$'' or ``Y:$<$option$>$'' respectively.

